% !TEX TS-program = pdflatex
% !TEX encoding = UTF-8 Unicode

\documentclass[11pt]{article} % use larger type; default would be 10pt
\usepackage[utf8]{inputenc} % set input encoding (not needed with XeLaTeX)
\usepackage{clays_notes}

\newcommand\R[1]{\mathbb{R}^{#1}}
\newcommand\C[1]{\mathbf{C^{#1}}}
\newcommand\F[1]{\mathbf{F^{#1}}}
\newcommand\U{\mathbf{U}}
\renewcommand\S{\mathcal{S}}
\renewcommand\u{\mathbf{u}}
\newcommand\V{\mathbf{V}}
\newcommand\W{\mathbf{W}}


% FONT STYLES

\title{Linear Algebra Done Right \\ Axler, Sheldon}
\author{Notes by:  \\ Clay Curry}
\date{}

\begin{document}

\section{Preliminaries}

Abstract algebra is the study of the structures, also known as \textbf{Algebraic Structures}, provided by operations (mappings between elements) on sets. Of these structures, a traditional first course in Abstract Algebra covers the theoretical aspects of groups, rings, and fields. 

\noindent\rule{\textwidth}{1pt}

\subsubsection{Contents of MATH 4323}
These notes cover the contents of my Abstract Algebra I course (MATH 4323). This includes:\\
proofs, induction,  the division algorithm, congruences and symmetries, groups, subgroups, cyclic groups, repeated squaring, complex numbers, permutations, dihedral groups, cosets, Lagrange's Theorem, Fermat's and Euler's Theorems, private key cryptography, public key cryptography, isomorphisms, direct products, normal subgroups, simplicity of $A_n$, homomorphisms, isomorphism theorems, matrix groups, finite Abelian groups, solvable groups, group actions, the Class Equations, Bursinde's Counting Theorem, the Sylow Theorems, applications of the Sylow Theorems.

\noindent\rule{\textwidth}{1pt}

\subsection{Set Theory}
A basic knowledge of set theory, mathematical induction, equivalence relations, and matrices is a must. Even more important is the ability to read and understand mathematical proofs.

\subsubsection{A Short note on Proofs}

Although mathematics is often motivated by physical experimentation or by computer simulations, \textbf{mathematics is made rigorous through the use of logical arguments}.

\definition
{Axiomatic Approach to Abstract Mathematics}
{
In studying abstract mathematics, we take what is called an \textbf{axiomatic approach}; that is, we: 
\points{take a collection of objects $\S$, and} 
{assume some rules about their structure.} 
}

\noindent
Rules about the structure of elements in a set are called \textbf{axioms}. With a \textbf{set} and a \textbf{system of axioms}, we can achieve the following:
\points
{derive other information about S by using logical arguments}
{derive other information that does not contradict one another}
{not include redundant axioms}
{provide enough flexibility to investigate many kinds of mathematical objects}

\noindent
A \textbf{proposition} in logic or mathematics is an assertion that is either true or false. A \textbf{mathematical proof} is nothing more than a convincing argument about the accuracy of a proposition. Mathemeticians are interested in statements like ``if $p$, then $q$." Here p is
called the \textbf{hypothesis} and q is known as the \textbf{conclusion}. Many mathematical conclusions are derived from a chain of hypotheses and conclusions.

The following are useful facts about proofs. Never assume any hypothesis that is not explicitly stated in the theorem; you cannot take things for granted. To show that an object is unique, assume that there are two such objects, say r and s, and then show that r = s. Sometimes it is easier to prove the contrapositive of a statement. Although it is better to find a direct proof of a theorem, this task can sometimes be difficult. Use contradiction: assume the theorem is false and show that the assumption makes some statement that cannot possibly be true.

% PAGE 2
\subsubsection{Sets and Equivalence Relations}
\definition{Set}
{A \textbf{set} is a well-defined collection of objects; that is, it is defined in such a manner that we can determine for any given object $x$ whether or not $x$ belongs to the set.}

We can find various relations between sets as well as perform operations on sets.

\example{Subsets}
{$$
\mathbb{N} \subset \mathbb{Z} \subset \mathbb{Q} \subset \mathbb{R} \subset \mathbb{C}
$$}

When two sets have no elements in common, they are said to be \textbf{disjoint}; for example, if $E$ is the set of even integers and $O$ is the set of odd integers, then E and O are disjoint.

\definition{Difference of Sets}
{We define the difference of two sets A and B to be:
$$
A \\ B = A \cap B' = \{ x : x \in A, x \notin B\}
$$}

\subsection{Algebraic Structures}
\subsubsection{Operations}
\textbf{Operations} are a special kind of kind of function that map zero or more input values from a set $\S ^n$ to a unique element in $\S$. 
Operations on any collection of mathematical objects are the mathematical tools used to \textit{describe} the additional structure or relationships \textit{defined/discovered} between those objects. 

\example{}
{
The element $(1,2)\in \text{}< \text{}\subsetneq \R{2}$, because $<$ semantically depends on the ``ordering" structure on $\R{}$ that defines 1 to be ``less than" 2. 
}


\end{document}

















