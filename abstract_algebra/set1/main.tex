%%%%%%%%%%%%%%%%%%%%%%%%%%%%%%%%%%%%%%%%%
% fphw Assignment
% LaTeX Template
% Version 1.0 (27/04/2019)
%
% This template originates from:
% https://www.LaTeXTemplates.com
%
% Authors:
% Class by Felipe Portales-Oliva (f.portales.oliva@gmail.com) with template 
% content and modifications by Vel (vel@LaTeXTemplates.com)
%
% Template (this file) License:
% CC BY-NC-SA 3.0 (http://creativecommons.org/licenses/by-nc-sa/3.0/)
%
%%%%%%%%%%%%%%%%%%%%%%%%%%%%%%%%%%%%%%%%%

%----------------------------------------------------------------------------------------
%	PACKAGES AND OTHER DOCUMENT CONFIGURATIONS
%----------------------------------------------------------------------------------------

\documentclass[
	12pt, % Default font size, values between 10pt-12pt are allowed
	%letterpaper, % Uncomment for US letter paper size
	%spanish, % Uncomment for Spanish
]{fphw}

% Template-specific packages
\usepackage[utf8]{inputenc} % Required for inputting international characters
\usepackage[T1]{fontenc} % Output font encoding for international characters
\usepackage{mathpazo} % Use the Palatino font
\usepackage{booktabs} % Required for better horizontal rules in tables
\usepackage{enumerate} % To modify the enumerate environment
\usepackage{amsmath}
\usepackage{amssymb}
%----------------------------------------------------------------------------------------
%	MY SHORTCUTS
%----------------------------------------------------------------------------------------

\newcommand\set[1]{\{#1\}}
\newcommand\qed{\text{$\blacksquare$}}
\newcommand\R{\mathbb{R}}
\newcommand\Z{\mathbb{Z}}
\newcommand\N{\mathbb{N}}

%----------------------------------------------------------------------------------------

%----------------------------------------------------------------------------------------
%	ASSIGNMENT INFORMATION
%----------------------------------------------------------------------------------------

\title{Execise Set \#1} % Assignment title
\author{Clayton Curry} % Student name
\date{September 1st, 2021} % Due date
\institute{University of Oklahoma \\ Department of Mathematics} % Institute or school name
\class{Intro to Abstract Algebra (MATH 4323)} % Course or class name
\professor{Dr. Roi Docampo} % Professor or teacher in charge of the assignment

%----------------------------------------------------------------------------------------

\begin{document}

\maketitle % Output the assignment title, created automatically using the information in the custom commands above

%----------------------------------------------------------------------------------------
%	ASSIGNMENT CONTENT
%----------------------------------------------------------------------------------------

\section*{Question 1}

\begin{problem}
Suppose that
\begin{align*}
	A &= \set{x : x \in \mathbb{N} \text{ and x is even}}\\
	B &= \set{x : x \in \mathbb{N} \text{ and x is prime}}\\
	C &= \set{x : x \in \mathbb{N} \text{ and x is a multiple of 5}}
\end{align*}	
\end{problem}

%------------------------------------------------

\subsection*{Answer} Describe each of the following sets:\\
a) $A \cap B = \set{2}$   because 2 is the only even prime number\\
b) $B \cap C = \set{5}$ because a multiple of 5 cannot be prime\\
c) $A \cup B = \set{1, 2, 3, 4, 5, 6, 7, 8, 10, 11, 12, 13, 14, 16, 17, 18, 19, \ldots}$, the set of all naturals minus composite odd numbers\\
d) $A \cap (B \cup C)$ includes $2$ and the even multiples of $5$, because 2 is the only even prime.

%----------------------------------------------------------------------------------------

\section*{Question 2}

\begin{problem}
Suppose that $A = \set{a,b,c}, B = \set{1,2,3}, C = \set{x}$, and $D = \emptyset$.
\end{problem}

%------------------------------------------------

\subsection*{Answer} List all of the elements in each of the following sets\\
a) $A \times B = \set{(a,1),(a,2),(a,3),(b,1),(b,2),(b,3),(c,1),(c,2),(c,3)}$\\
b) $B \times A = \set{(1,a),(1,b),(1,c),(2,a),(2,b),(2,c),(3,a),(3,b),(3,c)}$\\
c) $A \times B \times C=\set{(a,1,x),(a,2,x),(a,3,x),(b,1,x),(b,2,x),(b,3,x),(c,1,x),(c,2,x),(c,3,x)}$\\
d) $A \times D = \emptyset$



%----------------------------------------------------------------------------------------
\section*{Question 8}

\begin{problem}
Prove $A \subset B$ if and only if $A \cap B = A$
\end{problem}

%------------------------------------------------

\subsection*{Answer} Take any two sets $B$ and $A \subset B$.
$$
A \subset B  \Longleftrightarrow \forall a \in A, a \in B \Longleftrightarrow A \cap B = A
$$
where the first iff holds by definition; the second iff holds because the set $A$ contains every element in $A$, $B$ contains at least every element in $A$, and their intersection contains no more elements than the collection defined by $A$.\qed

%----------------------------------------------------------------------------------------

\newpage

%----------------------------------------------------------------------------------------

\section*{Question 10}

\begin{problem}
Prove $A \cup B = (A \cap B) \cup (A \setminus B) \cup (B \setminus A)$
\end{problem}

%------------------------------------------------

\subsection*{Answer}
 Take any two sets $A, B \subset U$. By De Morgan's Laws and set opertation rules of distributivity,
\begin{align*}
A \cup B &= \set{x : x \in A} \cup \set{x : x \in B}\\
&=(A \cap U) \cup (B \cap U)\\
&=(A \cap (B \cup B')) \cup (B \cap (A \cup A'))\\
&=((A \cap B) \cup (A \cap B')) \cup ((B \cap A) \cup (B \cap A'))\\
&=((A \setminus B) \cup (A \setminus B')) \cup ((B \setminus A) \cup (B \setminus A'))\\
&=((A \setminus B) \cup (A \cap B)) \cup ((B \setminus A) \cup (B \cap A))\\
&=(A \cap B) \cup (B \cap A) \cup (A \setminus B) \cup (B \setminus A)\\
&=(A \cap B) \cup (A \setminus B) \cup (B \setminus A)\\
\end{align*}
as required. \qed
%----------------------------------------------------------------------------------------

\section*{Question 11}

\begin{problem}
Prove $(A \cup B) \times C = (A \times C) \cup (B \times C)$
\end{problem}

%------------------------------------------------

\subsection*{Answer}
 Take any three sets $A, B, C$.
\begin{align*}
(A \cup B) \times C &= \set{(x, c) : x \in (A \cup B), c \in C}\\
&= \set{(x, c) : x \in A, c \in C} \cup \set{(x, c) : x \in B, c \in C} \\
&=(A \times C) \cup (B \times C)
\end{align*}
as required. \qed
%----------------------------------------------------------------------------------------

\section*{Question 14}

\begin{problem}
Prove $A \setminus (B \cup C) = (A \setminus B) \cap (A \setminus C)$
\end{problem}

%------------------------------------------------

\subsection*{Answer}
 Take any three sets $A, B, C$.  By De Morgan's Laws and set opertation rules of distributivity,
\begin{align*}
A \setminus (B \cup C) &= A \cap (B \cup C)'\\
&= A \cap A \cap (B' \cap C')\\
&= A \cap B' \cap A \cap C'\\
&= (A \setminus B) \cap (A \setminus C)
\end{align*}
as required. \qed
%----------------------------------------------------------------------------------------

\newpage

%----------------------------------------------------------------------------------------

\section*{Question 16}

\begin{problem}
Prove $(A \setminus B) \cup (B \setminus A) = (A \cup B) \setminus (A \cap B)$
\end{problem}

%------------------------------------------------

\subsection*{Answer}
Take any two sets $A, B \subset U$. By De Morgan's Laws and set opertation rules of distributivity,
\begin{align*}
(A \setminus B) \cup (B \setminus A) &= (A \cap B') \cup (B \cap A')\\
&= ((A \cap B') \cup B) \cap ((A \cap B') \cup A')\\
&= ((A \cup B) \cap (B \cup B')) \cap ((A \cup A') \cap (B' \cup A'))\\
&= ((A \cup B) \cap U) \cap (U \cap (B' \cup A'))\\
&=(A \cup B) \cap (B' \cup A')\\
&= (A \cup B) \cap (A \cap B)'\\
&= (A \cup B) \setminus (A \cap B)
\end{align*}
as required. \qed
%----------------------------------------------------------------------------------------

\section*{Question 18}

\begin{problem}
Determine which of the following functions are one-to-one and which are onto. If the function is not onto, determine its range.
\end{problem}

%------------------------------------------------

\subsection*{Answer}
Functions:\\
a) $f : \R \to \R : x \mapsto e^x$ \\
This function is one-to-one but not onto. Its range is $\R _{>0}$\\
b) $f : \Z \to \Z : n \mapsto n^2 + 3$ \\
This function is not one-to-one or onto. Its range is $\Z_{\ge3}$\\
c) $f : \R \to \R : x \mapsto \sin(x)$\\
This function is not one-to-one or onto. Its range is $[-1, 1]$.\\
d) $f : \Z \to \Z : n \mapsto n^2$ \\
This function is not one to one or onto. Its range is $\set{0} \cup \N$.\\
%----------------------------------------------------------------------------------------

\section*{Question 19}

\begin{problem}
Let $f: A \to B$ and $g: B \to C$ be invertable mappings; that is, mappings such that $f^{-1}$ and $g^{-1}$ exist. Show that $(g \circ f)^{-1} = f^{-1} \circ g^{-1}$ 
\end{problem}

%------------------------------------------------

\subsection*{Answer}
Take any $a \in A$, and suppose $a\overset{f}{\mapsto}b\overset{g}{\mapsto}c$. Therefore $(g \circ f)(a) = c$. Then
$$
a = f^{-1}(b) = f^{-1}(g^{-1}(c)) = (f^{-1} \circ g^{-1})(c)
$$
Hence $(g \circ f)^{-1} = (f^{-1} \circ g^{-1})$, which was to be demonstrated. \qed

%----------------------------------------------------------------------------------------
\newpage
%----------------------------------------------------------------------------------------

\section*{Question 20}

\begin{problem}
Define a function $f : \N \to \N$ that is one-to-one but not onto.
Define a function $f : \N \to \N$ that is not one-to-one but onto.
\end{problem}

%------------------------------------------------

\subsection*{Answer}
Suppose $f : \N \to \N$.\\
a) The function defined by $f : x \mapsto x + 1$ is one-to-one but not onto.\\
b) The function defined by $f = (1,1) \cup \set{(x, x-1) : x \in \N \text{ and } x \ge 2}$ is onto but not one-to-one.

%----------------------------------------------------------------------------------------

\section*{Question 22b}

\begin{problem}
Let  $f: A \to B$ and $g: B \to C$ be maps. If $g \circ f$ is onto, show that $g$ is onto.
\end{problem}

%------------------------------------------------

\subsection*{Answer}
If $f: A \to B$, and $g \circ f$ is onto, then,
$$f(A) \subset B \implies g(f(A)) \subset g(B) \subset C \implies C \subset g(B) \subset C$$
where the first $\implies$ holds because the preimage of $g(B)$ contains the preimage of $g(f(A))$, and the second $\implies$ holds because $(g \circ f) (A) = g(f(A)) = C$, by definition of onto. This demonstrates $g(B) = C$. Therefore, the mapping $g$ is onto. \qed

%----------------------------------------------------------------------------------------

\section*{Question 22c}

\begin{problem}
Let  $f: A \to B$ and $g: B \to C$ be maps. If $g \circ f$ is one-to-one, show that $f$ is one-to-one.
\end{problem}

%------------------------------------------------

\subsection*{Answer}
Take any one-to-one composition, $g \circ f$. Since $g \circ f$ is one-to-one, it is invertible (the inverse was shown earlier to be $f^{-1} \circ g^{-1}$). Therefore
\begin{align*}
(g \circ f)(a) &= (g \circ f)(b)\\
g^{-1}(g \circ f)(a) &= g^{-1}(g \circ f)(b)\\
f(a) &= f(b)\\
 f^{-1}f(a)&=f^{-1}f(b)\\
a &= b
\end{align*}
where we see that $f(a)=f(b) \implies a=b$. Hence, $f$ is injective. \qed

%----------------------------------------------------------------------------------------
\newpage
%----------------------------------------------------------------------------------------

\section*{Question 24}

\begin{problem}
Let  $f: X \to Y$ be a map with $A_1, A_2 \subset X$ and $B_1, B_2 \subset Y$.
\end{problem}

%------------------------------------------------

\subsection*{Answer}Prove the following statements\\
a) $$f (A_1 \cup A_2) = f(A_1) \cup f(A_2)$$
Let $f: X \to Y$ be a map with $A_1, A_2 \subset X$. Then,
\begin{align*}
b \in f (A_1 \cup A_2) &\Longleftrightarrow \exists a \text{ such that } b = f(a), \text{ where } a \in A_1 \cup A_2\\
&\Longleftrightarrow a \in A_1 \text{ or } a \in A_2\\
&\Longleftrightarrow b \in f(A_1) \text{ or } b \in f(A_2)\\
&\Longleftrightarrow b \in f(A_1) \cup f(A_2)
\end{align*}
where the first $\Leftrightarrow$ restates the asssumption; the second $\Leftrightarrow$ holds by definition; the third $\Leftrightarrow$ holds because the preimage of $b$ belongs to $A_1$ or $A_2$; and the final $\Leftrightarrow$ reiterates the preceding statement. Since $b \in f (A_1 \cup A_2) \Leftrightarrow b \in f(A_1) \cup f(A_2)$, the equality $f (A_1 \cup A_2) = f(A_1) \cup f(A_2)$ is demonstrated.  \qed

\vspace{10 pt}

\noindent
b) $$f(A_1 \cap A_2) \subset f(A_1) \cap f(A_2)$$
Consider any $b \in f(A_1 \cap A_2)$.
\begin{align*}
\exists a \in A_1, A_2 \text { such that } b = f(a) &\implies b \in f(A_1) \text{ and } b \in f(A_2)\\
&\implies b \in f(A_1) \cap f(A_2)
\end{align*}
where the first $\implies$ holds because the preimage of $b$ must exist in both $A_1$ and $A_2$, and the second $\implies$ holds by definition. Since $b \in f(A_1 \cap A_2) \implies b \in f(A_1) \cap f(A_2)$, the relation $f(A_1 \cap A_2) \subset f(A_1) \cap f(A_2)$ is demonstrated. \qed

For an example that disproves the equality, consider the mapping $f : \R \to \R$ defined by $f(x) = x^2$ for all $x \in \R$. Define the subsets,
\begin{align*}
\R &\supset A_1 = [-1, 0]\\
\R &\supset A_2 = [0, 1]
\end{align*}
Notice,
$$f(A_1 \cap A_2) = f(\set{0}) = \set{0} \ne [0, 1] = [0, 1] \cap [0, 1] = f(A_1) \cap f(A_2)$$
Hence 
$$f(A_1 \cap A_2) \ne f(A_1) \cap f(A_2).$$

\newpage

\noindent
c)$$f^{-1}(B_1 \cup B_2) = f^{-1}(B_1) \cup f^{-1}(B_2)$$
Let the mapping $g : Y \to X$ be defined by the equation $g(b) = f^{-1}(b)$ for all $b \in B$. \\
In part a), 
$$
g(B_1 \cup B_2)=g(B_1) \cup g(B_2)
$$
was demonstrated for any arbitrary mapping. Hence,
$$f^{-1}(B_1 \cup B_2) = f^{-1}(B_1) \cup f^{-1}(B_2),$$
as required. \qed

\vspace{20 pt}
\noindent
d)$$f^{-1}(B_1 \cap B_2) = f^{-1}(B_1) \cap f^{-1}(B_2)$$
Notice, in part b),
$$f^{-1}(B_1 \cap B_2) \subset f^{-1}(B_1) \cap f^{-1}(B_2)$$
was demonstrated for any arbitrary mapping. 

The right-hand set will now be shown to contain the left-hand set. Suppose $f(A_1) = B_1$ and $f(A_2) = B_2$. Then for all $b \in B_1 \cap B_2$, $\exists a \in f^{-1}(B_1 \cap B_2)$ such that $b = f(a)$. Hence $(f^{-1}(B_1) \cap f^{-1}(B_2)) \subset f^{-1}(B_1 \cap B_2)$,  as required. \qed

\vspace{20 pt}
\noindent
e)$$f^{-1}(Y \setminus B_1) = X \setminus f^{-1}(B_1)$$

\noindent
In part d), it was shown that
$$f^{-1}(Y \setminus B_1) = f^{-1}(Y \cap B_1') = f^{-1}(Y) \cap f^{-1}(B_1')$$
Since $f$ is invertible, $f$ is bijective, hence $f^{-1}(Y) = X$. Since $f^{-1}(B_1')$ contains all the elements that do not map to $B_1$, $f^{-1}(B_1') = f^{-1}(B_1)'$, which is every element that does not map to $B_1$. Hence
$$
f^{-1}(Y \setminus B_1) = f^{-1}(Y) \cap f^{-1}(B_1') = X \cap f^{-1}(B_1)' = X \setminus f^{-1}(B_1)
$$
as required. \qed
%----------------------------------------------------------------------------------------

\end{document}
