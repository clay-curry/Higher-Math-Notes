%%%%%%%%%%%%%%%%%%%%%%%%%%%%%%%%%%%%%%%%%
% fphw Assignment
% LaTeX Template
% Version 1.0 (27/04/2019)
%
% This template originates from:
% https://www.LaTeXTemplates.com
%
% Authors:
% Class by Felipe Portales-Oliva (f.portales.oliva@gmail.com) with template 
% content and modifications by Vel (vel@LaTeXTemplates.com)
%
% Template (this file) License:
% CC BY-NC-SA 3.0 (http://creativecommons.org/licenses/by-nc-sa/3.0/)
%
%%%%%%%%%%%%%%%%%%%%%%%%%%%%%%%%%%%%%%%%%

%----------------------------------------------------------------------------------------
%	PACKAGES AND OTHER DOCUMENT CONFIGURATIONS
%----------------------------------------------------------------------------------------

\documentclass[
	12pt, % Default font size, values between 10pt-12pt are allowed
	%letterpaper, % Uncomment for US letter paper size
	%spanish, % Uncomment for Spanish
]{fphw}

% Template-specific packages
\usepackage[utf8]{inputenc} % Required for inputting international characters
\usepackage[T1]{fontenc} % Output font encoding for international characters
\usepackage{mathpazo} % Use the Palatino font
\usepackage{booktabs} % Required for better horizontal rules in tables
\usepackage{enumerate} % To modify the enumerate environment
\usepackage{amsmath}
\usepackage{amssymb}
%----------------------------------------------------------------------------------------
%	MY SHORTCUTS
%----------------------------------------------------------------------------------------

\newcommand\set[1]{\{#1\}}
\newcommand\qed{\text{$\blacksquare$}}
\newcommand\R{\mathbb{R}}
\newcommand\Z{\mathbb{Z}}
\newcommand\N{\mathbb{N}}
\newcommand*\quot[2]{{^{\textstyle #1}\big/ _{\textstyle #2}}}
%----------------------------------------------------------------------------------------

%----------------------------------------------------------------------------------------
%	ASSIGNMENT INFORMATION
%----------------------------------------------------------------------------------------

\title{Execise Set \#2} % Assignment title
\author{Clayton Curry} % Student name
\date{September 10th, 2021} % Due date
\institute{University of Oklahoma \\ Department of Mathematics} % Institute or school name
\class{Intro to Abstract Algebra (MATH 4323)} % Course or class name
\professor{Dr. Roi Docampo} % Professor or teacher in charge of the assignment

%----------------------------------------------------------------------------------------

\begin{document}

\maketitle % Output the assignment title, created automatically using the information in the custom commands above

%----------------------------------------------------------------------------------------
%	ASSIGNMENT CONTENT
%----------------------------------------------------------------------------------------

\section*{1.4 \#25}

\begin{problem}
Determine whether or not the following relations are equivalence
relations on the given set. If the relation is an equivalence relation,
describe the partition given by it. If the relation is not an equivalence state why it fails to be one.
\begin{align*}
&a) x \sim y \text{ in } \R{} \text{ if } x \ge y &&c) x \sim y \text{ in } \R{} \text{ if } |x - y| \le 4\\
&b) m \sim n \text{ in } \Z{} \text{ if } mn > 0 &&d) m \sim n \text{ in }  \Z{} \text{ if } m \equiv n \mod 6
\end{align*}	

\end{problem}

%------------------------------------------------

\subsection*{Answer} .\\
a) Because the $\ge$ relation does not possess symmetry, $1 \sim 0$ but $0 \not \sim 1$. Hence, $\sim$ is not an equivalence relation.\\
b) Because $0 \cdot 0 = 0 \not > 0$, $\sim$ does not possess reflexivity so $\sim$ is not an equivalence relation.\\
c) Let $x = 0, y = 3,$ and $z = 6$, notice $x \sim y$ and $y \sim z$ but $x \not \sim z$.  Therefore, $\sim$ does not possess transitivity so it is not an equivalence relation.\\
d) This is an equivalence relation where $\quot{\Z{}}{\sim}$ contains six equivalence classes, $[0], [1], $ $[2], [3], [4], [5]$.
%----------------------------------------------------------------------------------------

\section*{1.4 \#28}

\begin{problem}
Find the error in the following argument by providing a counterexample. “The reflexive property is redundant in the axioms for an
equivalence relation. If $x \sim y$, then $y \sim x$ by the symmetric property. Using the transitive property, we can deduce that $x \sim x$.”
\end{problem}

%------------------------------------------------

\subsection*{Answer} Take $R$ to be the binary relation on $X = \set{a, b}$ such that $R = \set{(b,  b)}$. Therefore, $R$ possesses symmetry and transitivity but not reflexivity, and the quotient set $\quot{X}{R} = \set{\set{b}}$ makes a subset of $2^X$ that is not parition of $X$. Hence $R$ lacks one of the most crucial properies of equivalence relations, because $R$ does not generate a set of equivalence classes that partition $X$.


%----------------------------------------------------------------------------------------
\newpage
\section*{2.3 \#1}

\begin{problem}
Prove that
$$
\sum_{j = 1}^n j^2 = \frac{n(n+1)(2n+1)}{6}
$$
for $n \in \N{}$.
\end{problem}

%------------------------------------------------

\subsection*{Answer} By induction, it will be demonstrated that for all $n \in \N{}$,
\begin{equation} \label{eq1}
\sum_{j = 1}^n j^2 = \frac{n(n+1)(2n+1)}{6}
\end{equation}
\textbf{Base case:} When $n = 1$, the left side of $(1)$ is $1^2 = 1$, and the right side is $(1)(2)(3)/6 = 1$, so both sides are equal and $(1)$ is true for $n=1$.\\
\textbf{Induction step:} Let $k \in \N{}$ be given and suppose $(1)$ is true for $n = k$. Then,
\begin{align*}
\sum_{j = 1}^{k+1} j^2 &= \sum_{j = 1}^{k} j^2 + (k + 1)^2\\
&=\frac{k(k+1)(2k+1)}{6} + \frac{6(k + 1)(k + 1)}{6}\\
&=\frac{(k+1)(k(2k+1)+6(k+1))}{6}\\
&=\frac{(k+1)(2k^2+7k + 6)}{6}\\
&=\frac{(k+1)(k+2)(2k+3)}{6}\\
&=\frac{(k+1)((k+1)+1)(2(k+1) + 1)}{6}
\end{align*}
Thus $(1)$ holds for $n = k + 1$, and the proof of the induction step is complete. \qed
%----------------------------------------------------------------------------------------
\newpage
\section*{2.3 \#3}

\begin{problem}
Prove that
$$
n! > 2^n
$$
for $n \ge 4$.
\end{problem}

%------------------------------------------------

\subsection*{Answer} By induction, it will be demonstrated that for all $\N{} \ni n \ge 4$,
\begin{equation} \label{eq1}
n! > 2^n
\end{equation}
\textbf{Base case:} When $n = 4$, the left side of $(2)$ is $(4)(3)(2)(1) = 24$, and the right side is $(2)(2)(2)(2) = 16$, so the factorial is larger than the exponential and $(2)$ is true for $n=4$.\\
\textbf{Induction step:} Let $k \in \N{}$ be given and suppose $(2)$ is true for $n = k$. Then for $k \ge 4$,
\begin{align*}
2^{k+1} &= 2 \cdot 2^k\\
&< 2(k!)\\
&< (k+1)(k!)\\
&= (k+1)!
\end{align*}
Thus $(2)$ holds for $n = k + 1$ when $k \ge 4$, and the proof of the induction step is complete. \qed

%----------------------------------------------------------------------------------------

\section*{2.3 \#5}

\begin{problem}
Prove that $10^{n+1} + 10^n + 1$ is divisible by 3 for $n \in \N{}$.
\end{problem}

%------------------------------------------------

\subsection*{Answer} By induction, it will be demonstrated that for all $n \in \N{}$,
\begin{equation}
10^{n+1} + 10^n + 1
\end{equation}
is divisible by 3.\\
\textbf{Base case:} When $n = 1$,
$$
10^{1+1} + 10^1 + 1 = 111 = 37 \cdot 3
$$
is divisible by 3, and the proof of the base case is complete. \\
\textbf{Induction step:} Let $k \in \N{}$ be given and suppose $(3)$ is divisible for $n = k$. Then for $n = k+1$,
\begin{align*}
10^{(k+1)+1} + 10^{k+1} + 1 &= 10(10^{k+1} + 10^k) + (10 - 9) \\
&=10(10^{k+1} + 10^k + 1) - 9 \\
&= 10 (3j) - 3 \cdot 3 \hspace{20 pt} \text{ for some $j \in \Z{}$, by assumption}\\
&= (10j - 3)\cdot 3
\end{align*}
which is divisible by 3, and the proof of the induction step is complete. \qed

%----------------------------------------------------------------------------------------

\section*{2.3 \#10}

\begin{problem}
Prove that
$$
\sum^n_{k = 1} \frac{1}{k(k+1)} = \frac{n}{n+1}
$$
for $n \in N{}$.
\end{problem}

%------------------------------------------------

\subsection*{Answer} By induction, it will be demonstrated that for all $n \in \N{}$,
\begin{equation}
\sum^n_{k = 1} \frac{1}{k(k+1)} = \frac{n}{n+1}
\end{equation}
\textbf{Base case:} When $n = 1$, the left hand side of $(4)$ equals
$$
\sum^1_{k = 1} \frac{1}{k(k+1)} =  \frac{1}{1(1+1)} = \frac{1}{1+1} = \frac{1}{2}
$$
and the right hand side of $(4)$ equals
$$
\frac{1}{1+1} = \frac{1}{2}
$$
and the proof of the base case is complete.\\
\textbf{Induction step:} Let $j \in \N{}$ be given and suppose $(4)$ holds for $n = j$. Then for $n = j+1$,
\begin{align*}
\sum^{j+1}_{k = 1} \frac{1}{k(k+1)} &= \sum^{j}_{k = 1} \frac{1}{k(k+1)} + \frac{1}{(j+1)(j+2)}\\
&=  \frac{j}{j+1} +  \frac{1}{(j+1)(j+2)}\\
&= \frac{j(j+2) + 1}{(j+1)(j+2)}\\
&= \frac{j^2 + 2j + 1}{j+1}{j+2}\\
&= \frac{j+1}{j+2}\\
&= \frac{(j+1)}{(j+1)+1}
\end{align*}
and the proof of the induction step is complete. \qed

%----------------------------------------------------------------------------------------
\newpage
\section*{2.3 \#12}

\begin{problem}
Let X be a set. Define the power set of X, denoted P(X), to be the set of all subsets of X. For every positive integer $n$, show that a set with exactly $n$ elements has a power set with exactly $2^n$ elements.

\end{problem}

%------------------------------------------------

\subsection*{Answer} By induction, it will be demonstrated that
\begin{equation}
|P(X)| = 2^{|X|}
\end{equation}
for any set $X$.\\
\textbf{Base case:} Let $X_n \subset U$ be $n$ elements, where $x_i$ denotes the $i$'th element, from $U$. When $n = 1$, the left hand side of $(5)$ equals
$$
| P(X) | = |\set{\emptyset, \set{x_1}}| = 2,
$$
and the right hand side of $(5)$ equals 
$$P(X)=2^{|\set{x_1}|} = 2^1 = 2,$$
and the base case holds.\\
\textbf{Inductive step:}
Let $k \in \N{}$ be given and suppose holds for $n = k$. Then
\begin{align*}
|P(X \cup \set{x_{k + 1}})| &= |P(\set{x_1, \ldots, x_k, x_{k + 1}})|\\
&= | P(X) | + | \set{ S \cup \set{x_{k + 1}} : S  \in P(X)}|\\
&= | P(X) | + | P(X) | \\
&= 2 | P(X) |\\
&= 2 \cdot 2^k\\
&= 2^{k+1}
\end{align*}
demonstrates that $(5)$ also holds for $n = k + 1$, and the proof of the induction step is complete. \qed
%----------------------------------------------------------------------------------------
\newpage
\section*{2.3 \#15 b \& e}

\begin{problem}
For each of the following pairs of numbers $a$ and $b$, calculate gcd$(a, b)$
and find integers $r$ and $s$ such that gcd$(a, b) = ra + sb$.\end{problem}

%------------------------------------------------

\subsection*{Answer} .\\
b) 234 and 165
\begin{align*}
234 &= 165 \cdot 1 + 69 &\implies& &165 &= 69 \cdot 2 + 27 &\implies& &69 &= 27 \cdot 2 + 15 \\
27 &= 15 \cdot 1 + 12 &\implies& &15 &= 12 \cdot 1 + 3 &\implies&&12 &= 3 \cdot 4 + 0
\end{align*}
The Euclidean Algorithm derives $\gcd (234,165) = 3$. \\
Working backwards,
\begin{align*}
3 &= 15 + (-1)12\\
&= 15 + (-1)(27 + (-1)15)\\
&= (2)15 + (-1)27\\
&= (2)(69 + (-2)27) + (-1)27\\
&= (2)69 + (-5)27\\
&= (2)69 + (-5)(165 + (-2)69)\\
&= (12)69 + (-5)165\\
&= (12)(234 + (-1)165) + (-5)165\\
&= (12)234 + (-17)165
\end{align*}
hence $r = 12$ and $s = 165$.

\vspace{15 pt}
\noindent
e) 23771 and 19945
\begin{align*}
23771 &= 19945 \cdot 1 + 3826 &\implies& &19945 &= 3826 \cdot  5 + 815 &\implies& &3826&= 815 \cdot 4 + 566\\
815 &= 566 \cdot 1 + 249 &\implies& &566 &= 249 \cdot 2 + 68 &\implies& &68 &= 45 \cdot 1 + 23\\
45 &= 23 \cdot 1 + 22 &\implies& &23 &= 22 \cdot 1 + 1 &\implies& &22 &= 1 \cdot 22 + 0
\end{align*}
The Euclidean Algorithm derives $\gcd (23771, 19945) = 1$.
Working backwards,
\begin{align*}
1 &= 23 + (-1)22 &&= 23 + (-1)(45+(-1)23)\\
&=(2)23 + (-1)45 &&= (2)(68 + (-1)45) + (-1)45\\
&=(2)68 + (-3)45 &&= (2)68 + (-3)(249 + (-3)68)\\
&=(11)68 + (-3)249 &&= (11)(566+(-2)249) + (-3)249\\
&=(11)566 + (-25)249 &&= (11)566 + (-25)(815 +(-1)566)\\
&=(36)566 + (-25)815 &&= (36)(3826 + (-4)815) + (-25)815\\
&=(36)3826 + (-169)815 &&=(36)3826 + (-169)(19945 + (-5)3826)\\
&=(881)3826 + (-169)19945 &&= (881)(23771 + (-1)19945) + (-169)19945\\
&=(881)23771 + (-1050)19945
\end{align*}
hence $r = 881$ and $s = -1050$.
%----------------------------------------------------------------------------------------
\end{document}
