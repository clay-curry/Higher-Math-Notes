% !TEX TS-program = pdflatex
% !TEX encoding = UTF-8 Unicode

\documentclass[11pt]{article} % use larger type; default would be 10pt
\usepackage[utf8]{inputenc} % set input encoding (not needed with XeLaTeX)
\usepackage{clays_notes}



% FONT STYLES

\title{Abstract Algebra \\ Judson, Thomas J.}
\author{Notes by:  \\ Clay Curry}
\date{}

\begin{document}
\maketitle
\stepcounter{section}
\stepcounter{section}
\section{Groups}

One of the richest algebraic structures is one in which sets have a single operation that generalizes structures like $\Z{}$ together with its single operation of addition and the set of invertible $2 \times 2$ matrices together with its single operation of matrix multiplication. These sets, together with their respective operations, are examples of algebraic structures known as \textbf{groups}.

Historically, the theory of groups arose from an attempt to find the roots of a polynomial in terms of its coefficients. Nowadays, modern group theory is uesd in the study of coding theory, counting, and symmetries: a deep philosophical concept touching almost every area of scientific study including biology, chemistry, and physics.

\subsection{Intger Equivalence Classes}

This section provides examples of mathematical structures that can be viewed as sets with single operations.

\subsubsection{The Integers mod n}
\definition{Modulus n}
{Fix $n \in \N{}$ and $a,b \in \Z{}$. The \textbf{modulo $\mathbf{n}$ operation} relates $a$ and $b$ when $n$ divides $a-b$. Here $n$ is said to be the \textbf{modulus} of the mod $n$ relation.  The following statements are equivalent
\points
{$n | a - b$}
{``$a$ and $b$ are congruent modulo $n$"}
{$a \equiv b \mod n$}
{$a \equiv_n b$}
The integers mod $n$ relation partitions $\Z{}$ into $n$ different equivalence classes; the these equivalence classes are described in notation by:
\points
{$\Z{}_n$ for the combined $n$ equivalence classes (also called congruence classes)  in $^{\Z{}} / _{\equiv_n}$ }
{$[a]_n := \set{x \in \Z{}: x \equiv a \mod n}$ for the individual congruence classes.}
}

Note the exists a 1-to-1 correspondence between conguence classes and the set of remainders.

\example{Applications of Integers mod n}
{\points{Cryptograpy}{Coding Theory}{Detection of Errors in Identification Codes}}

\newpage
It is possible to define arithmetic operations on $\Z{}_n$; for any two integers, 
\definition{Arithmetic on $\Z{}_n$}
{Define
Addition modulo $n$ by the following formula 
$$[a + b]_n = [a]_n + [b]_n$$
Multiplication modulo $n$ by the following formula
$$[ab]_n = ([a]_n)([b]_n)$$

\theorem{Properties of modular aritmetic}
{Every non-zero $[k]_n \in Z_n$ has an inverse in $Z_n$ if $k$ is relatively prime to $n$. That is, $k$ has an inverse in $\Z{}_n$ if
$$
\gcd(k, n) = 1
$$}
{}

}

\subsection{Symmetries}
A map from the plane to itself preserving the symmetry of an object is called a \textbf{rigid motion}. A \textbf{symmetry} of a rectangle include all the rigid motions of the rectangle: the identity motion, $180^\circ$ rotation, reflection across verticle axis, and reflection across the horizontal axis.

\definition{Symmetry}
{A \textbf{symmetry} of a geometric figure is a rearrangement of the figure preserving the arrangement of its sides and vertices as well as its distances and angles.}

\newpage
\subsubsection{Definitions and Examples}
A \textbf{binary operation} or \textbf{law of composition} on a set $G$ is a function $G \times G \to G$ that assigns to each pair $(a, b) \in G \times G$ a unique element $a \circ b$ or $ab$ in $G$, called the \textit{composition of} $a$ and $b$. 
\definition{Group}
{A \textbf{group} $(G, \circ)$ is a set $G$ together with a law of composition $(a,b) \mapsto a \circ b$ that satisfies the following axioms.
\points
{The law of composition is \textbf{associative}. That is for all $a,b,c \in G$.
$$a \circ (b \circ c) = (a \circ b) \circ c$$}
{There exists an element $e \in G$, called the \textbf{ identity element} in G, such that for any $a \in G$
$$e \circ a = a \circ e = a$$}
{For each element $a \in G$, there exists an \textbf{inverse element} in $G$, denoted by $a^{-1}$, such that
$$a \circ a^{-1} = a^{-1} \circ a = e$$}
}
A group $G$ with the property that $a \circ b = b \circ a$ for all $a,b \in G$ is called \textbf{abelian} or \textbf{commutative}. Groups not satisfying this property are said to be \textbf{nonabelian} or \textbf{noncommutative}.

It is often convenient to describe a group in terms of an addition or multiplication table. Such a table is called a \textbf{Cayley table}.

\begin{center}
\noindent\begin{tabular}{c | c c c c c}
    + & 0 & 1 & 2 & 3 & 4  \\
    \cline{1-6}
    0 & 0 & 1 & 2 & 3 & 4 \\
    1 & 1 & 2 & 3 & 4 & 0 \\
    2 & 2 & 3 & 4 & 0 & 1 \\
    3 & 3 & 4 & 0 & 1 & 2 \\
    4 & 4 & 0 & 1 & 2 & 3 \\
\end{tabular}

\vspace{10pt}
Cayley table for $(\Z{}_5, +)$
\end{center}

One property of modular multiplication was that every non-zero $[k]_n \in \Z{}_n$ has an inverse in $\Z{}_n$ if $k$ is relatively prime to $n$. \definition{Group of Units}
{The set of congruence classes in $\Z{}_n$ where the congruence class has a multiplicative inverse in $\Z{}_n$, denoted by $\mathbf{U(n)}$, is called the \textbf{group of units} of $\Z{}_n$.

\begin{center}
\noindent\begin{tabular}{c | c c c c c}
    $\cdot$ & 1 & 3 & 5 & 7\\
    \cline{1-5}
    1 & 1 & 3 & 5 & 7 \\
    3 & 3 & 1 & 7 & 5 \\
    5 & 5 & 7 & 1 & 3 \\
    7 & 7 & 5 & 3 & 1 \\
\end{tabular}

\vspace{10pt}
Cayley table for $U(8)$.
\end{center}
}

The \textbf{order} of a finite group is the number of elements that it contains. A group is \textbf{finite}, or has \textbf{finite order}, if it contains a finite number of elements; otehrwise the group is said to be \textbf{infinite} or to have \textbf{infinite order}.

\newpage
\subsubsection{Basic Properties of Groups}

\theorem{Unique identity element}
{The identity element in a group $G$ is unique; that is, there exists only one $e \in G$ such that $e \circ g = g \circ e = g$ for all $g \in G$.}
{}
\theorem{Unique inverse for each element}
{If $g$ is any element in a group $G$, then the inverse of $g$, denoted by $g^{-1}$, is unique.}
{}
\theorem{Inverse of a composition is unique}
{Let $G$ be a group. If $a, b \in G$ then $(ab)^{-1} = b^{-1} a^{-1}$.}
{}
\theorem{Inverse undoes inverse}
{Let $G$ be a group. For any $a \in G, (a^{-1})^{-1} = a$.}
{}


\begin{center}
\noindent\begin{tabular}{c | c c c c c}
    + & 0 & 1 & 2 & 3 & 4  \\
    \cline{1-6}
    0 & 0 & 1 & 2 & 3 & 4 \\
    1 & 1 & 2 & 3 & 4 & 0 \\
    2 & 2 & 3 & 4 & 0 & 1 \\
    3 & 3 & 4 & 0 & 1 & 2 \\
    4 & 4 & 0 & 1 & 2 & 3 \\
\end{tabular}

\vspace{10pt}
Cayley table for Sudoku.
\end{center}
\end{document}

















