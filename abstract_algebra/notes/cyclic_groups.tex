% !TEX TS-program = pdflatex
% !TEX encoding = UTF-8 Unicode

\documentclass[11pt]{article} % use larger type; default would be 10pt
\usepackage[utf8]{inputenc} % set input encoding (not needed with XeLaTeX)
\usepackage{clays_notes}

\newcommand\br{\vspace{15 pt}}
\newcommand\R[1]{\text{$\mathbb{R}^{#1}$}}
\newcommand\F[1]{\text{$\mathbb{F}^{#1}$}}
\newcommand\C[1]{\text{$\mathbb{C}^{#1}$}}
\newcommand\Z[1]{\text{$\mathbb{Z}^{#1}$}}
\newcommand\N[1]{\text{$\mathbb{N}^{#1}$}}
\newcommand\Q[1]{\text{$\mathbb{Q}^{#1}$}}
\newcommand\cyc[1]{\text{$\langle {#1} \rangle$}}
\renewcommand\P{\text{$\mathcal{P}$}}

\newcommand\U{\mathbf{U}}
\renewcommand\S{\mathcal{S}}
\renewcommand\u{\mathbf{u}}
\newcommand\set[1]{\{#1\}}
\newcommand\ls[2]{#1_1, \ldots, #1_{#2}}


% FONT STYLES

\title{Abstract Algebra \\ Judson, Thomas J.}
\author{Notes by:  \\ Clay Curry}
\date{}

\begin{document}
\maketitle
\setcounter{section}{3}
\section{Cyclic Groups}

The groups $\Z{}$ and $\Z{}_n$ are both examples of what are called cyclic groups. In this chapter, we will study the properites of cyclic groups and cyclic subgroups, which play a fundamental part in the classification of all abelian groups.

\subsection{Cyclic Subgroups}

% EXAMPLE 1
\example{$3\Z{}$}
{
Suppose we consider $3 \in \Z{}$ and look at all the multiples of $3$. As a set, this is
$$
3 \Z{} = \set{\ldots, -3, 0, 3, 6, \ldots}
$$
Here $3\Z{}$ is a subgroup of the integers, and this subgroup is completely determined by the element $3$ since we can obtain all of the other elements of the group my taking multiples of $3$. Every element in the group is said to be \textbf{generated} by $3$.
}

\br

% EXAMPLE 2
\example{$2^n$}
{
The following set and law of composition $(H, \cdot)$ is a subgroup of non-zero rationals \Q{*}
$$
H = \set{2^{n} : n \in \Z{}}
$$
because $H \ne \emptyset$ and for all $2^n,2^m \in H$, we have $2^n (2^{m})^{-1} = 2^n 2^{-m} = 2^{n-m} \in H$. Therefore, $H$ is a subgroup of \Q{*} determined by $2$. 
}

\br

% THEOREM 1
\theorem{Repeated composition forms a subgroup}
{
Let $G$ be a group and $a \in G$. Then the set
$$
\cyc{a} = \set{a^k : k \in \Z{}}
$$
is a subgroup of $G$. Furthermore, $\cyc{a}$ is the smallest subgroup of $G$ containing $a$.
}
{
If $a \in G$, then $a^0 = e \in G$ so $\cyc{a} \ne \emptyset$. Also if $m, n \in \Z{}$ and $a^{m+n}=a^m a^n \in \cyc a$, then $a^{m-n} = a^m a^{-n} \in \cyc{a}$, concluding that \cyc a is a subgroup of $\Q{*}$.

Since any group containing $a$ must contain all the powers $a^k$ of $a$, \cyc a is a subgroup of all groups containing $a$. Therefore, \cyc a is the smallest subgroup of $G$ containing $a$.
}

% DEFINITION - Cyclic Subgroup
\definition{Cyclic Groups}
{
For $a \in G$, the set \cyc a is called the \textbf{cyclic subgroup} of $G$ generated by $a$. If $G$ contains some element of $a$ such that $G = \cyc a$, then $G$ is a \textbf{cyclic group}. In this case $a$ is called a \textbf{generator} of $G$.
}

\br

\definition{Order}
{
If $a$ is an element of a group $G$, we define the \textbf{order} of $a$ to be the smallest positive integer $n$ such that $a^n = e$, and we write $|a| = n$. If there is no such integer, we write $|a| = \infty$.
}

\br 
In general, a cyclic group can have more than a single generator. Both $1$ and $5$ generate $\Z{}_6$; hence, $\Z{}_6$ is a cyclic group. In general, not every element in a cyclic group is necessarily a generator of the group. The order of $2 \in \Z{}_6$ is 3.

In the symmetry group of an equaliateral triangle $S_3$, each element forms a distinct cyclic subgroup of $S_3$; however, no single element of $S_3$ generates $S_3$.

\br 
\theorem{}
{Every cyclic group is abelian}
{
Let $G$ be a cyclic group; therefore, $\exists a \in G$ such that $G = \cyc a$. Then for any two elements $g, h \in G$, since
$$
gh = a^n a^m = a^{n+m} = a^{m+n} = a^m a^n = hg
$$
G is abelian.
}

\br

\subsection{Subgroups of cyclic groups}
One interesting question we can asks about any group $G$ is, ``which subgroups of $G$ are cyclic?" Similarly, for any cyclic subgroup, we can ask, ``which subgroups for some cyclic subgroup $H \leqslant G$ are cyclic?"

\theorem{}
{Every subgroup of a cyclic group is cyclic}
{
Let $G$ be a cyclic group generated by $g \in G$ and suppose $H \leqslant G$. If $H = \set e$ then $e$ generates $H$. If some non-identity element $h = g^n \in H$, then $h^{-1} = g^{-n} \in H$, and either $n$ or $-n$ is positive.  Let $S = \set{ m \in \N{} : g^m \in H}$. Notice $n \in S$ or $-n \in S$, implying $\emptyset \ne S \subset \N{}$, which means $S$ has a least element $m = \min (S)$. Let $h' = g^m$

Claim: $\cyc {h'} = H$.\\
Since $H \leqslant G$ and $G$ is cyclic, $\forall h \in H, \exists k \in \Z{}$, such that $h = g^k$. By the division algorithm, $\exists q, r \in \Z{}$ where $0 \le r < m$, such that
$$
h = g^k = g^{mq+r} = (g^m)^q g^r \iff g^r = h (h')^{-q} \iff (g^r \in H) \wedge (r < m)
$$
and since $m = \min (S)$, we know $r = 0$. Because
$$
h = g^k = (g^m)^q = h'^q
$$
for all $h \in H$, $H$ is cyclic.
}

\br

\theorem{}
{Let $G$ be a cyclic group of order $n$ and suppose that $a$ is a generator for $G$. Then $a^k = e$ if and only if $n$ divides $k$.}
{}

\br

\theorem{}
{Let $G$ be a cyclic group of order $n$ and suppose that $a \in G$ is a generator of the group. If $b = a^k$, then the order of $b$ is $n/d$, where $d = \gcd (k,n)$.
}
{}
\end{document}