% !TEX TS-program = pdflatex
% !TEX encoding = UTF-8 Unicode

\documentclass[11pt]{article} % use larger type; default would be 10pt
\usepackage[utf8]{inputenc} % set input encoding (not needed with XeLaTeX)
\usepackage{clays_notes}

\newcommand\R[1]{\text{$\mathbb{R}^{#1}$}}
\newcommand\F[1]{\text{$\mathbb{F}^{#1}$}}
\newcommand\C[1]{\text{$\mathbb{C}^{#1}$}}
\newcommand\Z[1]{\text{$\mathbb{Z}^{#1}$}}
\newcommand\N[1]{\text{$\mathbb{N}^{#1}$}}
\renewcommand\P{\text{$\mathcal{P}$}}

\newcommand\U{\mathbf{U}}
\renewcommand\S{\mathcal{S}}
\renewcommand\u{\mathbf{u}}
\newcommand\set[1]{\{#1\}}
\newcommand\ls[2]{#1_1, \ldots, #1_{#2}}


% FONT STYLES

\title{Abstract Algebra \\ Judson, Thomas J.}
\author{Notes by:  \\ Clay Curry}
\date{}

\begin{document}
\maketitle
\stepcounter{section}
\section{The Integers}

\subsection{Induction}

\definition{First Principle of Mathematical Induction}
{Let $S(n)$ be a statement about integers for $n \in \N{}$ and suppose $S(n_0)$ is true for some integer $n_0$. If for all integers $k$ with $k \ge n_0$, $S(k)$ implies that $S(k+1)$ is true, then $S(n)$ is true for all integers $n$ greater than or equal to $n_0$.}

\definition{Second Principle of Mathematical Induction}
{Let $S(n)$ be a statement about integers for $n \in \N{}$ and suppose $S(n_0)$ is true for some integer $n_0$. If $S(n_0), S(n_0 + 1), \ldots, S(k)$ imply that $S(k + 1)$ for $k \ge n_0$, then the statement $S(n)$ is true for all integers $n \ge n_0$.}

\definition{Principle of Well-Ordering}
{Every non-empty subset of the natural numbers contains a least element.}

\theorem{}{The Principle of Mathematical Induction implies that 1 is the least natural number}{}

\theorem{}{The Principle of Mathematical Induction implies the Principle of Well-Ordering. That  is, every nonempty subset of $\N{}$ contains a least element.}{}

\newpage
\subsection{The Division Algorithm}

An application of the Principle of Well-Ordering that is often-used is the division algorithm.

\theorem{Division Algorithm}
{
Let $a$ and $b$ be integers, with $b \ge 0$. Then there exists unique integers $q$ and $r$ such that
$$
a = bq + r
$$
where $0 \le r < b$. 
}
{\textit{existence of $q$ and $r$}. Consider the set,
$$
R = \set{a - bx : x \in \Z{} \wedge a - bx \ge 0}
$$
If $0 \in R$, then $b | a$, and we can let $q = a/b$ and $r = 0$. If $0 \not \in R$, then the WOP guarentees the existence of a smallest element in a set $R$ iff $R \subseteq \N{}$ and $R \ne \emptyset$. Since $a-xb \in \Z{}$ and $a-xb > 0$, clearly $R \subset N$ the first condition of the WOP is satisfied. To show that $R \ne \emptyset$, consider the two cases:

\textbf{Case 1:} $a \ge 0$. By letting $x = 0$, clearly $a \in R$.

\textbf{Case 2:} $a < 0$. By letting $x = 2a$ and substituting $a-bx = a - b(2a) = a(1-2b)$, we have the product of two negative integers $a$ and $(1-2b)$, when $b > 1$ (which is being claimed), hence $a - bx \in R$.

\vspace{5 pt}
\noindent
Because $R \subset \N{}$ and $R \ne \emptyset$, the WOP guarentees that there exists a least element $r = a - bq$.
}

\end{document}

















