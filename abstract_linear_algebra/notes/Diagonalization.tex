% !TEX TS-program = pdflatex
% !TEX encoding = UTF-8 Unicode

\documentclass[11pt]{article} % use larger type; default would be 10pt
\usepackage[utf8]{inputenc} % set input encoding (not needed with XeLaTeX)
\usepackage{clays_notes}


\newcommand\A[1]{\text{$A_{#1}$}}
\newcommand\B[1]{\text{$\mathcal{B}_{#1}$}}
\newcommand\R[1]{\text{$\mathbb{R^{#1}}$}}
\newcommand\C[1]{\text{$\mathbb{C^{#1}}$}}
\newcommand\F[1]{\text{$\mathbb{F^{#1}}$}}
\newcommand\M[1]{\text{$\mathcal{M}(#1)$}}
\newcommand\V{\text{$V$}}
\newcommand\U{\text{$U$}}
\newcommand\W{\text{$W$}}
\newcommand\0{\text{$\mathbf{0}$}}
\renewcommand\L[2]{\mathcal{L}(#1,#2)}


\newcommand\lc[3]{#1_1 #2_1 + \cdots + #1_{#3} #2_{#3}}
\newcommand\ls[2]{#1_1, \ldots, #1_{#2}}
\newcommand\spann[1]{\mathbf{span}(#1)}
\newcommand\set[1]{\{#1\}}
\newcommand\poly[1]{\mathcal{P}_{#1}(\F{})}
\renewcommand\deg{\text{deg }}
\renewcommand\dim{\text{dim }}
\newcommand\nul{\text{null }}
\newcommand\range{\text{range }}

% FONT STYLES

\title{Eigenvectors}
\author{Notes by:  \\ Clay Curry}
\date{}

\begin{document}
\setcounter{section}{3}
\section{Eigentheory and Diagnolization}

In this chapter we develop the tools that will help us \textbf{understand the structure of vector space endomorphisms (linear operators)}, that is, maps from a vector space $\V \to \V$ that preserve the operations of addition and scalar multiplication over some field $\F{}$ that endow structure on $\V$. 

By focusing on certain subspaces, called \textbf{invariant subspaces} of the endomorphism, we will show that these so called \textbf{``eigenspaces"} allow us to create a simpler characterization of endomorphisms on $\V$.
\definition{Linear Endomorphisms}
{In the study of linear maps on vector spaces, an \textbf{endomorphism} is a structure preserving map from a vector space to itself, or \textbf{homomorphism} from a vector space to itself. For the rest of the class, we will focus on linear maps from a vector space to itself
$$T : \V \to \V$$
These are sometimes called \textbf{linear endomorphisms} of \V.
}
\noindent
Suppose $T$ is a linear endomorphism on $\V$. Notice that if we have a direct sum decomposition of $\V$
$$
V = \U_1 \oplus \U_2 \oplus \cdots \oplus \U_m
$$
where each $\U_j$ is a proper subspace of $\V$, then to understand the behavior of $T$, we need only to understand the behavior of \textbf{the maps of $T$ restricted to each subspace $\U_j$; each $T |_{\U_j}$}. The reason we can characterize all of $T$ by focusing on each $T |_{\U_j}$ is because any element of $\V$ is described by one unique linear combination of elements in $\U_j$.


\note{Notational Shortcuts}
{\points
{$\F{}$ denotes $\R{}$ or $\C{}$}
{$\V$ and $\W$ denote vector spaces over $\F{}$}
}
\subsection{Recall:}

\definition{Change of Basis Matrix}
{For any two bases $\B1,\B2$ of $\V$, the \textbf{change of basis matrix} from $\B1$ to $\B2$ is the matrix $[Id_\V]_{\B2,\B1}$}

\definition{Conjugation of matrices}
{Let $A, B$ be matrices. The \textbf{conjugation of $A$ by $B$} is the matrix $BAB^{-1}$. Therefore, the matrix of $T$ w.r.t. $\B2$ is the conjugation of the matrix of $T$
w.r.t. $\B1$ by the change of basis matrix from $\B1$ to $\B2$.}

\definition{Similarity}
{Two square matrices are \textbf{similar} if one is the conjugation of the other by some invertible matrix.}

\definition{Determinant}
{If $T: \V \to \V$ is linear and $\dim V < \infty$, the \textbf{determinant} of $T$ is $\det([T]_{\B{},\B{}})$, where $\B{}$ is any choice of basis for $\V$.}


\subsection{Invariant Subspaces}
Notice the above description of $T$ being characterized by $m$ restriction maps $T |_{\U_j}$ on proper subspaces of $\V$ requires $T$ to be an endomorphism on each $\U_j$. The notion of a subspace that is its own codomain under $T$ is sufficiently important to deserve a name.

\definition{Invariant Subspace}
{Suppose $T$ is a linear endomorphism on $\V$. A subspace $\U$ of $\V$ is called invariant under $T$ if $u \in \U$ implies $Tu \in \U$. That is, $\U$ is said to be an \textbf{invariant subspace of $\V$} if
$$
T(\U) \subseteq \U
$$} 

\noindent
In other words, $\U$ is invariant under $T$ if $T|_U$ is an endomorphism on $\U$.

Take any $v \in V$ with $v \ne0$ and let $\U$ equal the set of all scalar multiples of $v$. Because $U$ has dimension $1$, if $U$ is invariant under an endomorphism $T$, then $T$ acts as multiplication on $U$, and hence there is a scalar $\lambda \in \F{}$ such that
\begin{equation} \label{eq:1}
Tv = \lambda v
\end{equation}

Equation \ref{eq:1} is important enough that the vectors $v$ and scalars $\lambda$ satisfying it are given special names.
\definition{Eigenvalue}
{Suppose $T$ is a linear endomorphism on $\V$. A number $\lambda \in \F{}$ is called an \textbf{eigenvalue} of $T$ if there exists $v \in V$ such that $v \ne 0$ and $Tv = \lambda v$.}

\definition{Eigenvector}
{Suppose $T$ is an endomorphism on $\V$ and $\lambda \in \F{}$ is an eigenvalue of $T$. A vector $v \in V$ is called an \textbf{eigenvector} of $T$ cooresponding to $\lambda$ if $v \ne 0$ and $Tv = \lambda v$.}


\theorem{Equivalent Conditions to be an Eigenvalue}
{Suppose $\V$ is finite-dimensional, $T$ is an endomorphism on $\V$, and $\lambda \in \F{}$. Then the following are equivalent.
\points
{$\lambda$ is an eigenvalue of $T$ with $v$ the cooresponding eigenvector;}
{$(T - \lambda I_V)$ is not injective and $v \in \ker{(T - \lambda I_V)}$;}
{$(T - \lambda I_V)$ is not surjective;}
{$(T - \lambda I_V)$ is not invertable;}
Therefore, the eigenvectors of $T$ with eigenvalue $\lambda$ are the non-zero vectors in $\ker{(T-\lambda I_V)}$}{}

The eigenvectors associated with a particular eigenvalue is important enough that the set of all vectors associated with a particular eigenvalues is given a special name.
\definition{Eigenspace}
{Let $T$ be an endomorphism on $\V$ and $\lambda \in \F{}$. The $\mathbf{\lambda}$\textbf{-eigenspace of }$\mathbf{T}$ is
$$
\ker(T-\lambda I_V)
$$ 
}

\subsection{Characteristic Polynomials}

\end{document}