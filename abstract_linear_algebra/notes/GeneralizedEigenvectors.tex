% !TEX TS-program = pdflatex
% !TEX encoding = UTF-8 Unicode

\documentclass[11pt]{article} % use larger type; default would be 10pt
\usepackage[utf8]{inputenc} % set input encoding (not needed with XeLaTeX)
\usepackage{clays_notes}


\newcommand\A[1]{\text{$A_{#1}$}}
\newcommand\B[1]{\text{$B_{#1}$}}
\newcommand\R[1]{\text{$\mathbb{R^{#1}}$}}
\newcommand\C[1]{\text{$\mathbb{C^{#1}}$}}
\newcommand\F[1]{\text{$\mathbb{F^{#1}}$}}
\newcommand\M[1]{\text{$\mathcal{M}(#1)$}}
\newcommand\V{\text{$V$}}
\newcommand\U{\text{$U$}}
\newcommand\W{\text{$W$}}
\newcommand\0{\text{$\mathbf{0}$}}
\renewcommand\L[2]{\mathcal{L}(#1,#2)}


\newcommand\lc[3]{#1_1 #2_1 + \cdots + #1_{#3} #2_{#3}}
\newcommand\ls[2]{#1_1, \ldots, #1_{#2}}
\newcommand\spann[1]{\mathbf{span}(#1)}
\newcommand\set[1]{\{#1\}}
\newcommand\poly[1]{\mathcal{P}_{#1}(\C{})}
\renewcommand\deg{\text{deg }}
\renewcommand\dim{\text{dim}}
\renewcommand\null{\text{null }}
\newcommand\range{\text{range }}

% FONT STYLES

\title{Generalized Eigenvectors}
\author{Notes by:  \\ Clay Curry}
\date{}

\begin{document}

\section{Generalized Eigenvectors}
\subsection{Complexification}
Recall that a matrix $T$ is diagnolizable if and only if \textbf{the sum of the geometric multiplicities equals the dimension of $\V$.}
Recall that if there are the maximum number of distinct eigenvalues, then $T$ is diagonalizable.
\theorem{Enough roots of the characteristic polynomial}
{If $T$ is diagonalizable; that is,
$$
\sum_\lambda \text{geom.multi. of }\lambda = \dim(\V)
$$
then
$$
\sum_\lambda \text{alg.multi. of }\lambda = \dim(\V)
$$
that is, the characteristic polynomial $p_T$ factors as the product of linear terms.
}{}

If we consider the vector spaces over $\C{}$, we know that any characteristic polynomial factors as the product of linear terms by the Fundamental Theorem of Algebra
\theorem{Fundamental Theorem of Algebra}
{Every nonzero complex polynomial $p \in \poly{}$ can be factored, in essentially a unique way, as a product of a constant and linear terms, in the form
$$
p(x)=\prod a(x-\lambda)^{\mu(\lambda)}
$$
where $a$ is the leading coefficient of $p(x)$ and $\mu(\lambda)$ denotes the algebraic multiplicity of $\lambda$.}{}

These two facts can be combined to show the following
\theorem{Consequences for multiplicity}
{If $T: \V \to \V$ is $\C{}$-linear and $\dim_\C{}(\V) < \infty$, then
$$
\sum_\lambda \text{alg.multi. of }\lambda = \dim_\C{}(\V)
$$
also $T$ is diagonalizable iff for each eigenvalue $\lambda$ of $T$,
$$
\text{alg.multi. of }\lambda = \text{geom.multi. of }\lambda
$$

}{}



\end{document}