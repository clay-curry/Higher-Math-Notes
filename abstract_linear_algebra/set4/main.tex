%%%%%%%%%%%%%%%%%%%%%%%%%%%%%%%%%%%%%%%%%
% fphw Assignment
% LaTeX Template
% Version 1.0 (27/04/2019)
%
% This template originates from:
% https://www.LaTeXTemplates.com
%
% Authors:
% Class by Felipe Portales-Oliva (f.portales.oliva@gmail.com) with template 
% content and modifications by Vel (vel@LaTeXTemplates.com)
%
% Template (this file) License:
% CC BY-NC-SA 3.0 (http://creativecommons.org/licenses/by-nc-sa/3.0/)
%
%%%%%%%%%%%%%%%%%%%%%%%%%%%%%%%%%%%%%%%%%

%----------------------------------------------------------------------------------------
%	PACKAGES AND OTHER DOCUMENT CONFIGURATIONS
%----------------------------------------------------------------------------------------

\documentclass[
	12pt, % Default font size, values between 10pt-12pt are allowed
	%letterpaper, % Uncomment for US letter paper size
]{fphw}

% Template-specific packages
\usepackage[utf8]{inputenc} % Required for inputting international characters
\usepackage[T1]{fontenc} % Output font encoding for international characters
\usepackage{mathpazo} % Use the Palatino font
\usepackage{graphicx} % Required for including images
\usepackage{booktabs} % Required for better horizontal rules in tables
\usepackage{listings} % Required for insertion of code
\usepackage{enumerate} % To modify the enumerate environment

\usepackage{amsmath}
\usepackage{amssymb}
%----------------------------------------------------------------------------------------
%	MY SHORTCUTS
%----------------------------------------------------------------------------------------

\newcommand\0{\mathbf{0}}
\newcommand\set[1]{\{#1\}}
\newcommand\qed{\text{$\blacksquare$}}
\newcommand\R[1]{\text{$\mathbb{R}^{#1}$}}
\newcommand\F[1]{\text{$\mathbb{F}^{#1}$}}
\newcommand\Z{\text{$\mathbb{Z}$}}
\newcommand\N{\text{$\mathbb{N}$}}
\newcommand\U{\text{$U$ }}
\newcommand\ls[2]{\text{$#1_1, \ldots, #1_{#2}$}}
\newcommand\poly[1]{\text{$\mathcal{P}_{#1}(\F{})$ }}
\newcommand\spann[1]{\mathbf{span}(#1)}
\renewcommand\deg[1]{\textbf{deg$(#1)$}}
\renewcommand\dim[1]{\mathbf{dim}(#1)}
\newcommand\lc[3]{#1_1 #2_1 + \cdots + #1_{#3} #2_{#3}}
\renewcommand\L[2]{\mathcal{L}(#1, #2)}

%----------------------------------------------------------------------------------------

%----------------------------------------------------------------------------------------
%	ASSIGNMENT INFORMATION
%----------------------------------------------------------------------------------------

\title{Homework \#4} % Assignment title
\author{Clayton Curry} % Student name
\date{Sep 19, 2021} % Due date
\institute{University of Oklahoma \\ Department of Mathematics} % Institute or school name
\class{Abstract Linear Algebra} % Course or class name
\professor{Dr. Gregory Muller} % Professor or teacher in charge of the assignment

%----------------------------------------------------------------------------------------

\begin{document}

\maketitle % Output the assignment title, created automatically using the information in the custom commands above

%----------------------------------------------------------------------------------------
%	ASSIGNMENT CONTENT
%----------------------------------------------------------------------------------------

\section*{1C: 24}

\begin{problem}
A function $f : \R{} \to \R{}$ is called even if
$$
f(-x) = f (x)
$$
for all $x \in \R{}$. A function $f : \R{} \to \R{}$ is called odd if
$$
f(-x) = -f(x)
$$
for all $x \in \R{}$. 

Let $U_e$ denote the set of real-valued even functions on $\R{}$ and let $U_o$ denote the set of real-valued odd functions on $\R{}$. Show that
$$
\R{\R{}} = U_e \oplus U_o
$$
\end{problem}

%------------------------------------------------

\subsection*{Answer} 
If $f \in \R{\R{}}$, then
\begin{align*}
f(x) &= \frac{f(x) + f(x)}{2} + 0\\
&=\frac{f(x) + f(x)}{2} + \frac{f(-x) - f(-x)}{2}\\
&=\frac{f(x) + f(-x)}{2} + \frac{f(x) - f(-x)}{2}
\end{align*}
for all $x \in \R{}$. Notice, 
$$
\frac{f(x) + f(-x)}{2} \in U_e
$$
and
$$
 \frac{f(x) - f(-x)}{2} \in U_o
$$
which implies that any $f \in \R{\R{}}$ is the sum of an even and odd function, that is $\R{\R{}} = U_o + U_e$.  Another theorem states that the sum $\R{\R{}} = U_e + U_o$ is a direct sum if and only if $U_o \cap U_e = \set{0}$. This is easy to show. \qed

%----------------------------------------------------------------------------------------
\newpage
\section*{2C: 11}
\begin{problem}
Suppose that $U$ and $W$ are subspaces of $\R{8}$ such that $\dim{U} = 3$, $\dim{W} = 5$ and $U + W = \R{8}$. Prove that $\R{8} = U \oplus W$.
\end{problem}

\subsection*{Answer} 
Since $\dim{U} = 3$ and $\dim W = 5$, there exists three linearly independent spanning vectors $u_1, u_2, u_3$ in $U$ and five linearly independent spanning vectors $w_1,w_2$ $w_3,w_4,w_5$ in $W$. Since any spanning list of vectors in a subspace with length equal to the dimension of the subspace is a basis of the subspace, $u_1, u_2, u_3,w_1,w_2,w_3,w_4,w_5$ are necessarily a basis of $U + W$ and by extension linearly independent in $U + W$. Since $u_1, u_2, u_3,w_1,w_2,w_3,w_4,w_5$ are linearly independent, the only way to write $\0$ as a linear combination of $u$'s and $w$'s is by taking each coefficient to $\0$. Another theorem states that the sum of subspaces is a direct sum if and only if a linear combination equal to zero is the trivial one. Therefore, it then follows that $\R{8} = U \oplus W$.

%----------------------------------------------------------------------------------------

\section*{3A: 4}

\begin{problem}
Suppose $T \in \L V W$ and $\ls vm$ is a list of vectors in $V$ such that $\ls{Tv}m$ is linearly independent in $W$. Prove that $\ls vm$ is linearly independent.
\end{problem}

\subsection*{Answer}
Suppose $\ls cn \in \F{}$ where at least one $c_j \ne 0$ such that
$$
\0 = \lc cvn
$$
If $\ls{Tv}m$ is linearly independent in $W$, we have
$$
\0 = T(\lc c{v}n) = \lc c{Tv}n
$$
where at least one $c_j \ne 0$, which is a contratiction. More precisely, by assuming some set of independent vectors in $W$ under some $T \in \L VW$ is dependent in $V$, we have reached a contradiction. Therefore any set of independent vectors in $W$ under some $T \in \L VW$ must also be independent in $V$. \qed

%----------------------------------------------------------------------------------------

\section*{3A: 7}

\begin{problem}
Show that every linear map from a 1-dimensional vector space to itself is
multiplication by some scalar. More precisely, prove that if $\dim V = 1$
and $T \in \L{V,V}$ then there exists $\lambda \in \F{}$ such that $Tv = \lambda v$ for all $v \in V$.
\end{problem}

\subsection*{Answer} 
If $T \in \L VV$, then $T$ maps elements from $\set{ kv \in V : k \in \F{}, v \in V, v \ne 0}$ back to itself, because any non-zero $v \in V$ is a basis of $V$. By selecting the appropriate $\lambda \in \F{}$ such that $v\overset{T}{\mapsto}\lambda v$, then for any $kv \in V$, we have $T(kv) = k T(v) = k(\lambda v) = (k\lambda) v = (\lambda k) v = \lambda (kv)$. Since any element $kv$ maps to $\lambda (kv)$ under $T$, the transformation $T$ is indistinguishable from multiplication by $\lambda$.


%----------------------------------------------------------------------------------------
\newpage
\section*{3A: 14}

\begin{problem}
Suppose $V$ is finite-dimensional with $\dim V = 2$. Prove that there exist
$S, T \in \L VV$ such that $S T \ne T S$.
\end{problem}

\subsection*{Answer} 
Let $V = \R{2}$, and define the maps $T$ and $S$ by the following equations,
$$
T \begin{pmatrix}a \\ b\end{pmatrix} = \begin{pmatrix}1 & 2\\1 & 0\end{pmatrix}\begin{pmatrix}a \\ b\end{pmatrix} \hspace{15 pt}\text{and}\hspace{15 pt} S \begin{pmatrix}a \\ b\end{pmatrix} = \begin{pmatrix}2 & 1\\2 & 0\end{pmatrix}\begin{pmatrix}a \\ b\end{pmatrix}
$$
for all $\begin{pmatrix}a & b\end{pmatrix}^T \in \R{2}$. Notice $ST$ is defined by
$$
ST \begin{pmatrix}a \\ b\end{pmatrix} = \begin{pmatrix}2 & 1\\2 & 0\end{pmatrix} \begin{pmatrix}\begin{pmatrix}1 & 2\\1 & 0\end{pmatrix} \begin{pmatrix}a \\ b\end{pmatrix}\end{pmatrix} =  \begin{pmatrix}\begin{pmatrix}2 & 1\\2 & 0\end{pmatrix} \begin{pmatrix}1 & 2\\1 & 0\end{pmatrix}\end{pmatrix} \begin{pmatrix}a \\ b\end{pmatrix} = \begin{pmatrix}3&4\\2&4\end{pmatrix}\begin{pmatrix}a \\ b\end{pmatrix}
$$
and 
$$
TS \begin{pmatrix}a \\ b\end{pmatrix} = \begin{pmatrix}1 & 2\\1 & 0\end{pmatrix}\begin{pmatrix}\begin{pmatrix}2 & 1\\2 & 0\end{pmatrix}\begin{pmatrix}a \\ b\end{pmatrix}\end{pmatrix} = \begin{pmatrix}\begin{pmatrix}1 & 2\\1 & 0\end{pmatrix}\begin{pmatrix}2 & 1\\2 & 0\end{pmatrix}\end{pmatrix}\begin{pmatrix}a \\ b\end{pmatrix} = \begin{pmatrix}6 & 1 \\ 2 & 1\end{pmatrix} \begin{pmatrix}a \\ b\end{pmatrix}
$$
but
$$
\begin{pmatrix}3&4\\2&4\end{pmatrix}\begin{pmatrix}a \\ b\end{pmatrix} \ne \begin{pmatrix}6 & 1 \\ 2 & 1\end{pmatrix} \begin{pmatrix}a \\ b\end{pmatrix}
$$
for some $\begin{pmatrix}a & b\end{pmatrix}^T \in \R{2}$. Hence this example provides a case of $S,T$ with $ST \ne TS$. \qed

%----------------------------------------------------------------------------------------
\end{document}
