%%%%%%%%%%%%%%%%%%%%%%%%%%%%%%%%%%%%%%%%%
% fphw Assignment
% LaTeX Template
% Version 1.0 (27/04/2019)
%
% This template originates from:
% https://www.LaTeXTemplates.com
%
% Authors:
% Class by Felipe Portales-Oliva (f.portales.oliva@gmail.com) with template 
% content and modifications by Vel (vel@LaTeXTemplates.com)
%
% Template (this file) License:
% CC BY-NC-SA 3.0 (http://creativecommons.org/licenses/by-nc-sa/3.0/)
%
%%%%%%%%%%%%%%%%%%%%%%%%%%%%%%%%%%%%%%%%%

%----------------------------------------------------------------------------------------
%	PACKAGES AND OTHER DOCUMENT CONFIGURATIONS
%----------------------------------------------------------------------------------------

\documentclass[
	12pt, % Default font size, values between 10pt-12pt are allowed
	%letterpaper, % Uncomment for US letter paper size
]{fphw}

% Template-specific packages
\usepackage[utf8]{inputenc} % Required for inputting international characters
\usepackage[T1]{fontenc} % Output font encoding for international characters
\usepackage{mathpazo} % Use the Palatino font
\usepackage{graphicx} % Required for including images
\usepackage{booktabs} % Required for better horizontal rules in tables
\usepackage{listings} % Required for insertion of code
\usepackage{enumerate} % To modify the enumerate environment

\usepackage{amsmath}
\usepackage{amssymb}
%----------------------------------------------------------------------------------------
%	MY SHORTCUTS
%----------------------------------------------------------------------------------------

\newcommand\0{\mathbf{0}}
\newcommand\set[1]{\{#1\}}
\newcommand\qed{\text{$\blacksquare$}}
\newcommand\R[1]{\text{$\mathbb{R}^{#1}$}}
\newcommand\F[1]{\text{$\mathbb{F}^{#1}$}}
\newcommand\Z{\text{$\mathbb{Z}$}}
\newcommand\N{\text{$\mathbb{N}$}}
\newcommand\U{\text{$U$ }}
\newcommand\ls[2]{\text{$#1_1, \ldots, #1_{#2}$}}
\newcommand\poly[1]{\text{$\mathcal{P}_{#1}(\F{})$ }}
\newcommand\spann[1]{\mathbf{span}(#1)}
\renewcommand\deg[1]{\textbf{deg$(#1)$}}
\renewcommand\dim[1]{\mathbf{dim}(#1)}
\newcommand\lc[3]{#1_1 #2_1 + \cdots + #1_{#3} #2_{#3}}

%----------------------------------------------------------------------------------------

%----------------------------------------------------------------------------------------
%	ASSIGNMENT INFORMATION
%----------------------------------------------------------------------------------------

\title{Homework \#4} % Assignment title
\author{Clayton Curry} % Student name
\date{Sep 19, 2021} % Due date
\institute{University of Oklahoma \\ Department of Mathematics} % Institute or school name
\class{Abstract Linear Algebra} % Course or class name
\professor{Dr. Gregory Muller} % Professor or teacher in charge of the assignment

%----------------------------------------------------------------------------------------

\begin{document}

\maketitle % Output the assignment title, created automatically using the information in the custom commands above

%----------------------------------------------------------------------------------------
%	ASSIGNMENT CONTENT
%----------------------------------------------------------------------------------------

\section*{1C: 24}

\begin{problem}
A function $f : \R{} \to \R{}$ is called even if
$$
f(-x) = f (x)
$$
for all $x \in \R{}$. A function $f : \R{} \to \R{}$ is called odd if
$$
f(-x) = -f(x)
$$
for all $x \in \R{}$. 

Let $U_e$ denote the set of real-valued even functions on $\R{}$ and let $U_o$ denote the set of real-valued odd functions on $\R{}$. Show that
$$
\R{\R{}} = U_e \oplus U_o
$$
\end{problem}

%------------------------------------------------

\subsection*{Answer} 
If $f \in \R{\R{}}$, then
\begin{align*}
f(x) &= \frac{f(x) + f(x)}{2} + 0\\
&=\frac{f(x) + f(x)}{2} + \frac{f(-x) - f(-x)}{2}\\
&=\frac{f(x) + f(-x)}{2} + \frac{f(x) - f(-x)}{2}
\end{align*}
for all $x \in \R{}$. Notice, 
$$
\frac{f(x) + f(-x)}{2} \in U_e
$$
and
$$
 \frac{f(x) - f(-x)}{2} in U_o
$$
which implies that $f$ can be uniquely writted as the sum of an even function and an odd function. Since any $f \in \R{\R{}}$ can be expressed as a unique sum of an even function and odd function, $\R{\R{}}$ is a direct sum of $U_o$  and $U_e$. \qed

%----------------------------------------------------------------------------------------
\newpage
\section*{2C: 11}
\begin{problem}

\end{problem}

\subsection*{Answer} 
\qed

%----------------------------------------------------------------------------------------

\section*{3A: 4}

\begin{problem}

\end{problem}

\subsection*{Answer}
\qed

%----------------------------------------------------------------------------------------

\section*{3A: 7}

\begin{problem}

\end{problem}

\subsection*{Answer} 
\qed

%----------------------------------------------------------------------------------------

\section*{3A: 14}

\begin{problem}

\end{problem}

\subsection*{Answer} 
\qed

%----------------------------------------------------------------------------------------
\end{document}
