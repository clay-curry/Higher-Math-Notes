%%%%%%%%%%%%%%%%%%%%%%%%%%%%%%%%%%%%%%%%%
% fphw Assignment
% LaTeX Template
% Version 1.0 (27/04/2019)
%
% This template originates from:
% https://www.LaTeXTemplates.com
%
% Authors:
% Class by Felipe Portales-Oliva (f.portales.oliva@gmail.com) with template 
% content and modifications by Vel (vel@LaTeXTemplates.com)
%
% Template (this file) License:
% CC BY-NC-SA 3.0 (http://creativecommons.org/licenses/by-nc-sa/3.0/)
%
%%%%%%%%%%%%%%%%%%%%%%%%%%%%%%%%%%%%%%%%%

%----------------------------------------------------------------------------------------
%	PACKAGES AND OTHER DOCUMENT CONFIGURATIONS
%----------------------------------------------------------------------------------------

\documentclass[
	12pt, % Default font size, values between 10pt-12pt are allowed
	%letterpaper, % Uncomment for US letter paper size
	%spanish, % Uncomment for Spanish
]{fphw}

% Template-specific packages
\usepackage[utf8]{inputenc} % Required for inputting international characters
\usepackage[T1]{fontenc} % Output font encoding for international characters
\usepackage{mathpazo} % Use the Palatino font
\usepackage{graphicx} % Required for including images
\usepackage{booktabs} % Required for better horizontal rules in tables
\usepackage{listings} % Required for insertion of code
\usepackage{enumerate} % To modify the enumerate environment

\usepackage{amsmath}
\usepackage{amssymb}
%----------------------------------------------------------------------------------------
%	MY SHORTCUTS
%----------------------------------------------------------------------------------------

\newcommand\set[1]{\{#1\}}
\newcommand\qed{\text{$\blacksquare$}}
\newcommand\R[1]{\text{$\mathbb{R}^{#1}$}}
\newcommand\F[1]{\text{$\mathbb{F}^{#1}$}}
\newcommand\Z{\text{$\mathbb{Z}$}}
\newcommand\N{\text{$\mathbb{N}$}}
\newcommand\U{\text{$U$ }}
\newcommand\ls[2]{\text{$#1_1, \ldots, #1_{#2}$}}
\newcommand\poly[1]{\text{$\mathcal{P}_{#1}(\F{})$ }}
\newcommand\spann[1]{\mathbf{span}(#1)}
\renewcommand\deg[1]{\textbf{deg$(#1)$}}


%----------------------------------------------------------------------------------------

%----------------------------------------------------------------------------------------
%	ASSIGNMENT INFORMATION
%----------------------------------------------------------------------------------------

\title{Homework \#1} % Assignment title
\author{Clayton Curry} % Student name
\date{Sep 5, 2021} % Due date
\institute{University of Oklahoma \\ Department of Mathematics} % Institute or school name
\class{Abstract Linear Algebra} % Course or class name
\professor{Dr. Gregory Muller} % Professor or teacher in charge of the assignment

%----------------------------------------------------------------------------------------

\begin{document}

\maketitle % Output the assignment title, created automatically using the information in the custom commands above

%----------------------------------------------------------------------------------------
%	ASSIGNMENT CONTENT
%----------------------------------------------------------------------------------------

\section*{1C: 7}

\begin{problem}
Give an example of a nonempty subset \U of \R{2} such that \U is closed under addition and under taking additive inverses (meaning $-u \in U$ whenever $u \in \U$ ), but \U is not a subspace of \R{2}.
\end{problem}

%------------------------------------------------

\subsection*{Answer} 

The set of pairs of integers $U = \Z^2 \subset \R{2}$ contains the sum for any two elements in $\Z^2$, and $\Z^2$ contains an additive inverse for every element in $\Z^2$. However, $\Z^2$ is not closed under scalar multiplication since $\frac{1}{2}(1,0) = (\frac{1}{2},0) \not\in \Z^2$.

%----------------------------------------------------------------------------------------

\section*{1C: 8}

\begin{problem}
Give an example of a nonempty subset \U of \R{2} such that \U is closed under scalar multiplication, but \U is not a subspace of \R{2}.
\end{problem}

%------------------------------------------------

\subsection*{Answer} The set of pairs of reals of the same sign, that is
$$
U = \set{(x_1,x_2) \in \R{2} : \text{sgn}(x_1) = \text{sgn}(x_2)}
$$
is closed under scalar multiplication, but for any sufficiently small $\epsilon \in \R{}$, we have $(\epsilon,1) \in U$ and $(-1, -\epsilon) \in U$, but their sum $(\epsilon, 1) + (-1, -\epsilon) = (\epsilon + (-1), 1 + (-\epsilon)) \approx (-1, 1) \not \in U$.

%----------------------------------------------------------------------------------------
\newpage

\section*{1C: 9}

\begin{problem}
A function $f : \R{} \to \R{}$ is called periodic if there exists a positive number $p$ such that $f(x) = f(x + p)$ for all $x \in \R{}$. Is the set of periodic functions from \R{} to \R{} a subspace of \R{\R{}}? Explain.\end{problem}

%------------------------------------------------

\subsection*{Answer} Assume, for the sake of contradiction, that the sum of any two elements in $\U \subset \R{\R{}}$ is contained in $\U$. Let $f(x) = \sin(2\pi x)$ and $g(x) = \sin(2\pi x / \sqrt{2})$.  Therefore $f, g \in \U$ and $f$ has periods $\Z$ and $g$ has periods $\set{n\sqrt{2} : n \in \Z}$. 

Notice that the sum of these two functions are periodic, i.e. 
$$(f+g)(x) = f(x) + g(x) = f(x + p) + g(x + p) = (f+g)(x + p)$$
exactly when some non-zero $p \in \Z \cap \set{n\sqrt{2} : n \in \Z}$. In other words, $(f+g)$ is periodic when there exists at least one $k \in \Z$ and one $n \in \Z$, such that 

$$
p = k = n \sqrt{2}
$$
which implies,
$$\frac{k}{n} = \sqrt{2}$$
for some $k, n \in \Z$. But $\sqrt{2}$ is irrational, hence we have arrived at a contradiction by assuming the sum of two elements in $\U$ must be in $\U$. Therefore $\U$ is not a subspace.


%----------------------------------------------------------------------------------------
\newpage
\section*{2A: 1}

\begin{problem}
Suppose $v_1, v_2,  v_3, v_4$ spans $V$. Prove that the list
$$v_1-v_2, v_2 - v_3, v_3-v_4, v_4$$
also spans $V$.

\end{problem}

%------------------------------------------------

\subsection*{Answer} Consider any $v \in V$. Therefore $\exists$ scalars $a,b,c,d \in \F{}$ such that,
$$v = av_1 + bv_2 + cv_3 + dv_4$$
Notice, \begin{align*}
v_1 &= v_1-v_2 + v_2 - v_3 + v_3-v_4 + v_4\\
v_2 &= v_2 - v_3 + v_3-v_4 + v_4\\
v_3 &= v_3-v_4 + v_4\\
v_4 &= v_4
\end{align*}
Therefore,
$$
v = a(v_1-v_2) + (a+b)(v_2 - v_3) + (a+b+c)(v_3-v_4) + (a +b+c+d)v_4
$$
Since any $v \in V$ can be written as a linear comination of $v_1-v_2, v_2 - v_3, v_3-v_4, v_4$, these elements span $V$.


%----------------------------------------------------------------------------------------

\section*{2A: 3}

\begin{problem}
Find a number $t$ such that 
$$(3,1,4), (2,-3,5), (5,9,t)$$
is not linearly independent in \R{3}.

\end{problem}

%------------------------------------------------

\subsection*{Answer} Let $t=2$. Then\\
$$
0 = 3
\begin{pmatrix}
3 \\
1\\
4
\end{pmatrix}
-2
\begin{pmatrix}
2 \\
-3\\
5
\end{pmatrix}-
\begin{pmatrix}
5 \\
9\\
2
\end{pmatrix}$$
Therefore the vectors contain a non-trivial linear combination equal to zero and are not linearly independent.

%----------------------------------------------------------------------------------------

\section*{2A: 9}

\begin{problem}
Prove or give a counterexample: If \ls{v}{m} and \ls{w}{m} are linearly independent lists of vectors in $V$, then $v_1 + w_1, \ldots, v_m + w_m$ is linearly independent.

\end{problem}

%------------------------------------------------

\subsection*{Answer} Notice $(1, 0), (0, 1)$ is linearly independent and $(-1, 0), (0, -1)$ is linearly independent, but $(1, 0)+(-1, 0), (0, 1)+(0, -1) = (0,0),(0,0)$ is linearly dependent.\\


%----------------------------------------------------------------------------------------
\newpage

\section*{2B: 5}

\begin{problem}
Prove or disprove: there exists a basis $p_0, p_1, p_2, p_3$ of \poly{3} such that none of the polynomials $p_0, p_1, p_2, p_3$ has degree 2.

\end{problem}

%------------------------------------------------

\subsection*{Answer} Define $p_0, p_1, p_2, p_3 \in \poly{3}$ by the following equations,
\begin{align*}
p_0(x) &= x^3 + x^2\\
p_1(x) &= -x^3\\
p_2(x) &= x\\
p_3(x) &= 1
\end{align*}
for all $x \in \F{}$.
Notice,
\begin{align*}
\deg{p_0} &= 3\\
\deg{p_1} &= 3\\
\deg{p_2} &= 1\\
\deg{p_3} &= -\infty
\end{align*}
and for any element $p \in \poly{3}$ defined by the equation $p(x) = ax^3 + bx^2 + cx + d$,
$$
p(x) = (b)(x^3 + x^2) + (b-a)(-x^3) +  cx + d1
$$

%----------------------------------------------------------------------------------------


\section*{2A: 6}

\begin{problem}
Suppose $v_1, v_2, v_3, v_4$ is a basis of $V$. Prove that
$$
v_1 + v_2, v_2 + v_3, v_3 + v_4, v_4
$$
is a basis of $V$.
\end{problem}

%------------------------------------------------

\subsection*{Answer} Choose any $v \in \spann{v_1, v_2, v_3, v_4}$. Therefore, there exists a collection of scalars $a_1, a_2, a_3, a_4 \in \F{}$ such that,
$$
v = a_1v_1 + a_2v_2 + a_3v_3 + a_4v_4
$$
Notice
\begin{align*}
v &= a_1((v_1 + v_2) - (v_2 + v_3) + (v_3 + v_4) - v_4) \\
&+ a_2 ((v_2 + v_3) - (v_3 + v_4) + v_4) + a_3((v_3 + v_4) - v_4) + a_4(v_4)\\
&=a_1(v_1+v_2) + (a_2-a_1)(v_2+v_3) \\
&+ (a_1-a_2+a_3)(v_3 + v_4) + (a_2-a_1+a_4-a_3)(v_4).
\end{align*}
Since, any $v \in V$ can be uniquely expressed using $v_1 + v_2, v_2 + v_3, v_3 + v_4, v_4$, these are a basis for $V$.
%----------------------------------------------------------------------------------------


\end{document}
