%----------------------------------------------------------------------------------------
%	PACKAGES AND OTHER DOCUMENT CONFIGURATIONS
%----------------------------------------------------------------------------------------

\documentclass[
	12pt, % Default font size, values between 10pt-12pt are allowed
	%letterpaper, % Uncomment for US letter paper size
]{fphw}

% Template-specific packages
\usepackage{enumerate} % To modify the enumerate environment

\usepackage{amsmath}
\usepackage{amssymb}
%----------------------------------------------------------------------------------------
%	MY SHORTCUTS
%----------------------------------------------------------------------------------------

\newcommand\0{\mathbf{0}}
\newcommand\set[1]{\{#1\}}
\newcommand\qed{\text{$\blacksquare$}}
\newcommand\R[1]{\text{$\mathbb{R}^{#1}$}}
\newcommand\F[1]{\text{$\mathbb{F}^{#1}$}}
\newcommand\Z{\text{$\mathbb{Z}$}}
\newcommand\B[1]{\text{$\mathcal{B}_{#1}$}}
\newcommand\nulll[1]{\text{null($#1$)}}
\newcommand\N{\text{$\mathbb{N}$}}
\newcommand\U{\text{$U$ }}
\newcommand\ls[2]{\text{$#1_1, \ldots, #1_{#2}$}}
\newcommand\poly[1]{\text{$\mathcal{P}_{#1}(\R{})$}}
\newcommand\spann[1]{\mathbf{span}(#1)}
\renewcommand\deg[1]{\textbf{deg$(#1)$}}
\renewcommand\dim[1]{\mathbf{dim}(#1)}
\newcommand\lc[3]{#1_1 #2_1 + \cdots + #1_{#3} #2_{#3}}
\renewcommand\L[1]{\mathcal{L}(#1)}

%----------------------------------------------------------------------------------------

%----------------------------------------------------------------------------------------
%	ASSIGNMENT INFORMATION
%----------------------------------------------------------------------------------------

\title{Homework \#6} % Assignment title
\author{Clayton Curry} % Student name
\date{Oct 3, 2021} % Due date
\institute{University of Oklahoma \\ Department of Mathematics} % Institute or school name
\class{Abstract Linear Algebra} % Course or class name
\professor{Dr. Gregory Muller} % Professor or teacher in charge of the assignment

%----------------------------------------------------------------------------------------

\begin{document}

\maketitle % Output the assignment title, created automatically using the information in the custom commands above


%----------------------------------------------------------------------------------------
%	Problem 10.A: 2
%----------------------------------------------------------------------------------------
\section*{10.A: 2}
\begin{problem}
Suppose $A$ and $B$ are square matrices of the same size and $AB=I$. Prove that $BA=I$.
\end{problem}


\subsection*{Answer}
If $A,B$ are square matrices and $AB = I$, then the linear transformation $T_A$ is surjective, which implies $T_A$ is injective (because $A$ is a square matrix). Since $T_A$ is injective and surjective, $T_A$ is isomorphic and therefore $A^{-1}$ exists. Since $A^{-1}$ exists,
\begin{align*}
I &= A^{-1}A\\ 
&= A^{-1} I A\\
&= A^{-1}(AB)A\\
&=(A^{-1}A)BA\\
&=I(BA)\\
&=BA 
\end{align*}

%----------------------------------------------------------------------------------------
%	Problem 10.A: 3
%----------------------------------------------------------------------------------------
\section*{10A: 3}
\begin{problem}
Suppose $T \in \L{V}$ has the same matrix with respect to every basis of $V$. Prove that $T$ is a scalar multiple of the identity operator.
\end{problem}

%------------------------------------------------

\subsection*{Answer}
The following lemmas, $(1)$ and $(2)$, yield the following implications, demonstrating the above statement.
\begin{enumerate}
\item If there exists a non-zero vector in $V$ that is not an eigenvector of $T$, then there are two bases for $V$ such that the associated matrices of $T$ are not equal.
\item If every non-zero vector in $V$ is an eigenvector of $T$, then $T$ is a scalar multiple of the identity map.
\end{enumerate}
\begin{align*}
 \forall \text{ bases $\B1, \B2$ of } V, [T]_{\B1,\B1} = [T]_{\B2,\B2} &\overset{(1)}{\implies}  \forall v \in V, \exists \lambda \in \F{}, \text{ such that } Tv = \lambda v\\
&\overset{(2)}{\implies}\text{$T$ is a scalar multiple of the identity map.}
\end{align*}
Therefore, any linear endomorphism $T$ that has the same matrix with respect to every basis of $V$ is a scalar multiple of the identity operator on $V$.
\qed

\vspace{10 pt}
Proof of Lemma (1)\\
Define the linear endomorphism $T$ on $V$ by the equation $Tv_k = \lambda_kv_k$, where each $v_k$ belongs to an ordered basis $\B1=(\ls vn)$ of $V$, $\ls \lambda n \in \F{}$, and $\lambda_i \ne \lambda_j$. Since each $v_k \in \B1$ is an eigenvector of $T$, the matrix associated with $T$ for the basis $\B1$ is the diagonal matrix,
$$
[T]_{\B1, \B1}=
\begin{pmatrix}
\lambda_1 & 0 & \cdots & 0\\
0 & \lambda_2 & \cdots & 0\\
\vdots & \vdots & \ddots & \vdots \\
0 & 0 & \cdots & \lambda_n
\end{pmatrix}
$$

By extension, $\B2 = (v_1, \ldots, v_{n-1}, v_1 + \cdots +v_n)$ is a basis of $V$. Since $\not \exists \lambda \in \F{}$ where, $T v_k = \lambda v_k$ for all $v_k \in \B1$, $(v_1 + \cdots + v_n)$ is not an eigenvector of $T$ and the matrix $[T]_{\B2,\B2}$ cannot be diagonal. Since, there exists a non-zero vector in $V$ that is not an eigenvector of $T$, $[T]_{\B1,\B1} \ne [T]_{\B2,\B2}$.

\vspace{10 pt}
Proof of Lemma (2)\\
If, as assumed, any vector $v \in V$ is an eigenvector, then there exists a scalar map $\phi : V \to \F{}$, such that
\begin{equation}
Tv = (\phi v) v
\end{equation}
where $\phi v$ is the eigenvalue cooresponding to $v$. If it is shown that $\phi$ maps all $v \in V$ to the same element in $\F{}$, then it is demonstrated that $T$ is a scalar multiple of the identity map on $V$.

If $u, v \in V$ are related by the equation, $u = \beta v$ for some $\beta \in \F{}$, we have,
\begin{equation} \label{eq:2}
\phi(u)u = \phi(\beta v) \beta v = T(\beta v) = \beta T(v) = \beta \phi(v) v = \phi(v) \beta v
\end{equation}
When $(\beta v) \ne 0$, (\ref{eq:2}) shows that
\begin{equation} \label{eq:3}
\phi(\beta v) = \phi(v)
\end{equation}
i.e. $\phi$ maps every member of the one-dimensional subspace generated by $v$ to $\phi(v)$. When $u, v$ are linearly independent,
\begin{equation} \label{eq:4}
T(u + v) = \phi(u + v)(u + v) = \phi(u + v)u + \phi(u + v)v
\end{equation}
but additive property of $T$ also requires that,
\begin{equation} \label{eq:5}
T(u + v) = T(u) + T(v) = \phi(u)u + \phi(v)v
\end{equation}
combining (\ref{eq:4}) and (\ref{eq:5}),
\begin{equation} \label{eq:6}
(\phi(u + v) - \phi(u))u + (\phi(u + v)-\phi(v))v = 0
\end{equation}
equation (\ref{eq:6}) tells us $\phi(u + v) = \phi(u) = \phi(v)$ because we assumed that $u, v$ are linearly independent.

Since $u,v$ are an arbitrary pair of linearly independent vectors, the result produced by (\ref{eq:6}) combined with (\ref{eq:3}) shows that $\phi(v)$ is a constant for all $v \in V$; taking $\lambda = \phi(v)$ for any $v \in V$ then yields the complete solution. \qed
\newpage


%----------------------------------------------------------------------------------------
%	Problem 5.A: 2
%----------------------------------------------------------------------------------------
\section*{5.A: 2}
\begin{problem}
Suppose $S, T \in \L{V}$ are such that $ST = TS$. Prove that null$(S)$ is invariant under $T$.
\end{problem}
\subsection*{Answer}
Let $v \in \nulll S$. Then
\begin{align*}
S(Tv) &= T(Sv) = T(0) = 0\\ 
&\implies \forall v \in \nulll S, Tv \in \nulll S\\
&\implies T(\nulll S) \subseteq \nulll S
\end{align*}
Therefore null$(S)$ is invariant under $T$.

%----------------------------------------------------------------------------------------
%	Problem 5.A: 7
%----------------------------------------------------------------------------------------
\section*{5.A: 7}
\begin{problem}
Suppose $T \in \L{\R{2}}$ is defined by $T(x,y) = (-3y,x)$. Find the eigenvalues of $T$.
\end{problem}
\subsection*{Answer}
Suppose $\lambda$ is an eigenvalue of $T$. Then
\begin{align*}
\lambda x &= -3y\\
\lambda y &= x\\
0 &= (\lambda x + 3y) + (\lambda y - x)\\
(x-3y) &= \lambda(x+y)\\
\lambda &= \frac{(x-3y)}{(x+y)}
\end{align*}
Hence
$$
\lambda (1,0) = 1 (1, 0) = (1, 0)
$$
but
$$
T(1, 0) = (0,1) \ne \lambda (1,0)
$$
Therefore, no $\lambda \in \R{}$ is an eigenvalue of $T$


%----------------------------------------------------------------------------------------
%	Problem 5.A: 11
%----------------------------------------------------------------------------------------
\section*{5.A: 11}
\begin{problem}
Define $T: \poly{} \to \poly{}$ by $Tp = p'$. Find all eigenvalues and eigenvectors of $T$.
\end{problem}
\subsection*{Answer}
$\poly{} \subsetneq \mathcal{C}^\infty$ and the only function $f \in C^\infty$, such that $Tf = f' = cf$, is the function defined by $f(x) = e^{cx}$ for all $x \in \R{}$, but $e^{cx} \not \in \poly{}$. Therefore, no element in $\poly{}$ is an eigenvector for $T$ and there is no eigenvalue for $T$ in $\poly{}$.

\newpage

%----------------------------------------------------------------------------------------
%	Problem 5.A: 12
%----------------------------------------------------------------------------------------
\section*{5.A: 12}
\begin{problem}
Define $T \in \L{\poly{4}}$ by 
$$
(Tp)(x) = xp'(x)
$$
for all $x \in \R{}$. Find all eigenvalues and eigenvectors of $T$.
\end{problem}

\subsection*{Answer} Note that
\begin{align*}
T x^4 &= x(4x^3) = 4x^4& T x^3 &= x (3x^2) = 3x^3 \\
T x^2 &= x(2x) = 2x^2  & T x &= x (1) = 1\\
T(1) &= x (0) = 0
\end{align*}
Hence the span of each vector in the standard basis of $\poly{4}$ forms a distinct invariant subspace under $T$, with eigenvalue equal to the exact degree of the spanning polynomial.
\end{document}
