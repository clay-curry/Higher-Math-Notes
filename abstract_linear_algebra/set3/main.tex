%%%%%%%%%%%%%%%%%%%%%%%%%%%%%%%%%%%%%%%%%
% fphw Assignment
% LaTeX Template
% Version 1.0 (27/04/2019)
%
% This template originates from:
% https://www.LaTeXTemplates.com
%
% Authors:
% Class by Felipe Portales-Oliva (f.portales.oliva@gmail.com) with template 
% content and modifications by Vel (vel@LaTeXTemplates.com)
%
% Template (this file) License:
% CC BY-NC-SA 3.0 (http://creativecommons.org/licenses/by-nc-sa/3.0/)
%
%%%%%%%%%%%%%%%%%%%%%%%%%%%%%%%%%%%%%%%%%

%----------------------------------------------------------------------------------------
%	PACKAGES AND OTHER DOCUMENT CONFIGURATIONS
%----------------------------------------------------------------------------------------

\documentclass[
	12pt, % Default font size, values between 10pt-12pt are allowed
	%letterpaper, % Uncomment for US letter paper size
]{fphw}

% Template-specific packages
\usepackage[utf8]{inputenc} % Required for inputting international characters
\usepackage[T1]{fontenc} % Output font encoding for international characters
\usepackage{mathpazo} % Use the Palatino font
\usepackage{graphicx} % Required for including images
\usepackage{booktabs} % Required for better horizontal rules in tables
\usepackage{listings} % Required for insertion of code
\usepackage{enumerate} % To modify the enumerate environment

\usepackage{amsmath}
\usepackage{amssymb}
%----------------------------------------------------------------------------------------
%	MY SHORTCUTS
%----------------------------------------------------------------------------------------

\newcommand\0{\mathbf{0}}
\newcommand\set[1]{\{#1\}}
\newcommand\qed{\text{$\blacksquare$}}
\newcommand\R[1]{\text{$\mathbb{R}^{#1}$}}
\newcommand\F[1]{\text{$\mathbb{F}^{#1}$}}
\newcommand\Z{\text{$\mathbb{Z}$}}
\newcommand\N{\text{$\mathbb{N}$}}
\newcommand\U{\text{$U$ }}
\newcommand\ls[2]{\text{$#1_1, \ldots, #1_{#2}$}}
\newcommand\poly[1]{\text{$\mathcal{P}_{#1}(\F{})$ }}
\newcommand\spann[1]{\mathbf{span}(#1)}
\renewcommand\deg[1]{\textbf{deg$(#1)$}}
\renewcommand\dim[1]{\mathbf{dim}(#1)}
\newcommand\lc[3]{#1_1 #2_1 + \cdots + #1_{#3} #2_{#3}}

%----------------------------------------------------------------------------------------

%----------------------------------------------------------------------------------------
%	ASSIGNMENT INFORMATION
%----------------------------------------------------------------------------------------

\title{Homework \#3} % Assignment title
\author{Clayton Curry} % Student name
\date{Sep 12, 2021} % Due date
\institute{University of Oklahoma \\ Department of Mathematics} % Institute or school name
\class{Abstract Linear Algebra} % Course or class name
\professor{Dr. Gregory Muller} % Professor or teacher in charge of the assignment

%----------------------------------------------------------------------------------------

\begin{document}

\maketitle % Output the assignment title, created automatically using the information in the custom commands above

%----------------------------------------------------------------------------------------
%	ASSIGNMENT CONTENT
%----------------------------------------------------------------------------------------

\section*{2C: 1}

\begin{problem}
Suppose $V$ is finite-dimensional and $U$ is a subspace of $V$ such that
dim $U$ = dim $V$. Prove that $U$ = $V$.
\end{problem}

%------------------------------------------------

\subsection*{Answer} 
Suppose $\dim{U}$ = $\dim{V} = n$ and $U$ is a subspace of $V$. Then,
\begin{align*}
U \text{ is finite dimensional} &\implies \exists \text{ a basis } \ls{u}{n} \text{ of } U\\
&\implies \ls{u}{n} \text{ is linearly independent in } V
\end{align*}

Since every linearly independent list of vectors in $V$ of length $n$ is a basis of $V$, it must be true that $\ls{u}{n}$ is a basis of $V$. Therefore $U = \spann{\ls{u}{n}} = V$. \qed
%----------------------------------------------------------------------------------------
\newpage
\section*{2C: 7}
\begin{problem}
Let
$$
U = \set{p \in \poly{4} : p(2)=p(5)}.
$$
\end{problem}

%------------------------------------------------

\subsection*{Answer} .\\
(a) Find a basis of U.

Let $p \in \U{}$ be any polynomial such that $p(2) = p(5) = d$. Then $\exists a,b,c,d \in \F{}$ s.t.,
\begin{align}
p(x) &= (ax^2+bx + c)(x-2)(x-5) + d\\
&=ax^2(x-2)(x-5) + bx(x-2)(x-5) + c(x-2)(x-5) + d\cdot1\\
&=ax^4 + (-7a+b)x^3 +(10a-7b+c)x^2 + (10b + c)x + (10c+d)1
\end{align}

Hence $(2)$ describes all elements of the substructure $\poly{4}$ with a single linear combination. Additionally, by solving for $\0$ using the system of coefficients in $(3)$, 
\begin{equation}
p(x)=
\begin{pmatrix}
\begin{pmatrix}
1 & 0 & 0 & 0\\
-7& 1 & 0 & 0\\
10&-7 & 1 & 0\\
0 & 10 & -7 & 0\\
 0 & 0 & 10 & 1
\end{pmatrix}

\begin{pmatrix}
a\\b\\c\\d
\end{pmatrix}
\end{pmatrix}\cdot
\begin{pmatrix}
x^4\\x^3\\x^2\\x\\1
\end{pmatrix}
=
\begin{pmatrix}
0\\0\\0\\0\\0
\end{pmatrix}\cdot
\begin{pmatrix}
x^4\\x^3\\x^2\\x\\1
\end{pmatrix}
\end{equation}

it is clear that $p = \0 \iff a=b=c=d=0$. Therefore,
$$\mathcal{B}=\set{x^2(x-2)(x-5), x(x-2)(x-5), (x-2)(x-5), 1} $$ 

is a basis of $U$.\\

\noindent
(b) Extend the basis in part (a) to a basis of $\poly{4}$

By solving for $\begin{pmatrix}0 & 0 & 0 & 1 & 0 \end{pmatrix}^T$ using the matrix equation in $(4)$, it is clear that $x \not \in \spann{\mathcal{B}}$. Therefore, appending $x$ to $\mathcal{B}$ gives five linearly independent polynomials in $\poly{4}$. Consequentally, $\mathcal B \cup \set{x}$ is a basis of $\poly{4}$ because any $\dim{\poly{4}}$ linearly independent vectors in $\poly{4}$ generates $\poly{4}$.\\


\noindent
(c) Find a subspace $W$ of $\poly{4}$ such that $\poly{4} = U \oplus W$.

Let $W = \spann{x}$. According to $(b)$, when $p \in \poly{}$ is the mapping from $x\overset{p}{\mapsto}x$, we have $p \not \in U$. By the closure of scalar multiplication in $U$, we have $kp \not \in U$ for all $k \in \F{}$. Therefore, we know,
$$W \not \subset U \implies W \cap U = \set{0}$$
Part $(b)$ also concludes that 
$$\spann{\set{x^2(x-2)(x-5), x(x-2)(x-5), (x-2)(x-5), 1} \cup  \set{x}} = \poly{4}$$
By the definition of span, the previous equation implies
$$
\poly{4} = \spann{\set{x^2(x-2)(x-5), x(x-2)(x-5), (x-2)(x-5), 1}} + \spann{\set{x}} = U \oplus W
$$
%----------------------------------------------------------------------------------------

\newpage
\section*{2C: 10}

\begin{problem}
Suppose $p_0, \ls{p}{m} \in \poly{}$ are such that each $p_j$ has degree $j$ .
Prove that $p_0, \ls{p}{m}$ is a basis of $\poly{m}$.
\end{problem}

%------------------------------------------------

\subsection*{Answer} By strong induction it will be shown that $p_0, \ls{p}{m}$  is a basis of $\poly{m}$ where each $p_j$ has degree $j$.

\textbf{Base cases:} Let $m = 0$. Any $p_0, g_0 \in \poly{}$ where $\deg{p_0} = \deg{g_0} = 0$, say $p_0(x) = c$ and $g_0(x) = c'$, can be expressed in terms of each other by choosing $c'/c$ as a scaling factor for $p_0$, since
$$
\frac{c'}{c} p_0(x) = \frac{c'}{c} c = c' = g_0(x)
$$
Hence any $g \in \poly{0}$ can be expressed as a unique linear combination of a degree 0 polynomial.

Let $m = 1$.  For any $a'x + b' \in \poly{1}$, we have
\begin{align*}
a'x + b' = y_1 (ax + b) + y_0(c) \\
\iff y_1 = \frac{a'}{a}, y_0 = (b' - y_1b) = (b' - \frac{a'}{a}b)
\end{align*}
demonstrating that any $g \in \poly{1}$ can be expressed as a unique linear combination of a degree 1 polynomial and degree 0 polynomial. Hence $p_1, p_0$ is a basis of $\poly{1}$. 

Let $m = 2$.  For any $a'x^2 + b'x + c' \in \poly{2}$, we have
\begin{align*}
a'x^2 + b'x + c' = y_2 (a_{22}x^2 + a_{21}x +a_{20}) + y_1(a_{11}x + a_{10}) + y_0 (a_{00}) \\
\iff y_2 = \frac{a'}{a_{22}}, y_1 = \frac{1}{a_{11}}(b' - \frac{a'a_{21}}{a_{22}}), y_0 = \frac{1}{a_{00}}(c' - \frac{a'a_{20}}{a_{22}}-\frac{a_{10}}{a_{11}}(b' - \frac{a'a_{21}}{a_{22}}))
\end{align*}
demonstrating that any $g \in \poly{2}$ can be expressed as a unique linear combination of a degree 2 polynomial, degree 1 polynomial, and degree 0 polynomial. Therefore $p_2, p_1, p_0$ is a basis of $\poly{2}$.\\

\noindent
\textbf{Induction step:} Let $n = k$ be given and suppose $p_0, \ls{p}{k}$ is a basis for $\poly{k}$. Then any $g \in \poly{k+1}$ can be written as a linear combination of $p_0, \ls{p}{k}$ and a $p_{k+1}(x) = a_{(k+1)0} + a_{(k+1)1}x + \cdots + a_{(k+1)(k+1)}x^{k+1}$ term. More precisely, any $g$ in 
\begin{align*}
g(x) &= a_0 p_0(x) + a_1p_1(x) + \cdots + p_{k+1}(x)
\end{align*}
where $p_{k+1}(x)$ is defined immediately above, but we assumed, $$a_{(k+1)0} + a_{(k+1)1}x + \cdots + a_{(k+1)k}x^k \in \spann{p_0, \ls{p}{k}}$$ Using the process described in the proof of the linear dependence lemma, these redunant terms of $p_{k+1}$ can be removed until either every term in $p_{k+1}$ vanishes or until $g(x)$ can be expressed as a unique linear combination of $p_0, \ls{p}{k}$, and $x^{k+1}$. More precisely, any $g \in \poly{k+1}$ can be expressed as a unique linear combination of $p_0, \ls{p}{k+1}$ such that each $p_j$ has degree $j$. Therefore, the statement $p_0, \ls{p}{n}$ such that each $p_j$ has degree $j$ is a basis for $\poly{n}$ holds for $n = k+1$, and the inductive step is complete. 

%----------------------------------------------------------------------------------------
\newpage
\section*{2C: 12}

\begin{problem}
Suppose $U$ and $W$ are both five-dimensional subspaces of $\R{9}$. Prove
that $U \cap W \ne \set{0}$.
\end{problem}

%------------------------------------------------

\subsection*{Answer} If $U$ and $W$ are subspaces with $\dim{U} = \dim{W} = 5$, then there exists 5 linearly independet elements $\ls{u}{5} \in U \subset \R{9}$ and 5 linearly independent elements $\ls{w}{5} \in W \subset \R{9}$.  Assume $u_j \ne w_j$ for all $j \in \set{1,2,3,4,5}$.

Because $\R{9}$ is finite-dimensional, the cardinality of any linearly independent set in $\R{9}$ is less than or equal to the cardinality of a spanning set. Since $\ls{e}{9} \in \R{9}$ is a spanning set of \R{9}, 
$$
\ls{u}{5} \cup \ls{w}{5}
$$
cannot possibly be linearly independent. Therefore there exists a linear combination,
$$
\0 = \lc{a}{u}{5} + \lc{b}{w}{5}
$$
where $\ls{a}{5}, \ls{b}{5}$ are not all $\0$. By a theorem, if $U$ and $W$ are subspaces, then $U \cap W =\set{0}$ if and only if they are linearly independent. Since there exists a non-zero linear combination of elements of $U$ and $W$ equal to $\0$, the sets are not linearly independent. Therefore,
$$
U \cap W \ne \set{0}
$$
%----------------------------------------------------------------------------------------
\end{document}
