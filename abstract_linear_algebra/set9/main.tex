%----------------------------------------------------------------------------------------
%	PACKAGES AND OTHER DOCUMENT CONFIGURATIONS
%----------------------------------------------------------------------------------------

\documentclass[
	12pt, % Default font size, values between 10pt-12pt are allowed
	%letterpaper, % Uncomment for US letter paper size
]{fphw}

% Template-specific packages
\usepackage{enumerate} % To modify the enumerate environment

\usepackage{amsmath}
\usepackage{amssymb}
%----------------------------------------------------------------------------------------
%	MY SHORTCUTS
%----------------------------------------------------------------------------------------

\newcommand\br{\vspace{10 pt}}
\newcommand\0{\mathbf{0}}
\newcommand\set[1]{\{#1\}}
\newcommand\qed{\text{$\blacksquare$}}
\newcommand\R[1]{\text{$\mathbb{R}^{#1}$}}
\newcommand\C[1]{\text{$\mathbb{C}^{#1}$}}
\newcommand\F[1]{\text{$\mathbb{F}^{#1}$}}
\newcommand\Z{\text{$\mathbb{Z}$}}
\newcommand\B[1]{\text{$\mathcal{B}_{#1}$}}
\newcommand\nulll[1]{\text{null($#1$)}}
\newcommand\N{\text{$\mathbb{N}$}}
\newcommand\U{\text{$U$ }}
\newcommand\ls[2]{\text{$#1_1, \ldots, #1_{#2}$}}
\newcommand\poly[1]{\text{$\mathbb{P}_{#1}(\F{})$}}
\newcommand\spann[1]{\mathbf{span}(#1)}
\renewcommand\deg[1]{\textbf{deg$(#1)$}}
\renewcommand\dim[1]{\mathbf{dim}(#1)}
\newcommand\lc[3]{#1_1 #2_1 + \cdots + #1_{#3} #2_{#3}}
\renewcommand\L[1]{\mathcal{L}(#1)}
\newcommand\tr{\text{tr}}

%----------------------------------------------------------------------------------------

%----------------------------------------------------------------------------------------
%	ASSIGNMENT INFORMATION
%----------------------------------------------------------------------------------------

\title{Homework \#9} % Assignment title
\author{Clayton Curry} % Student name
\date{Oct 24, 2021} % Due date
\institute{University of Oklahoma \\ Department of Mathematics} % Institute or school name
\class{Abstract Linear Algebra} % Course or class name
\professor{Dr. Gregory Muller} % Professor or teacher in charge of the assignment

%----------------------------------------------------------------------------------------

\begin{document}
\maketitle % Output the assignment title, created automatically using the information in the custom commands above

%----------------------------------------------------------------------------------------
%	Problem 5
%----------------------------------------------------------------------------------------
\section*{Problem 5}
\begin{problem}
Suppose $T \in \L{V}, m$ is a positive integer, and $v \in V$ is such that
$T^{m-1} v \ne 0$ but $T^m v = 0$. Prove that
$$
v, T v, T^2 v, \cdots, T^{m-1}v
$$
is linearly independent.
\end{problem}

\subsection*{Answer} Because $T^m v = 0, T^{m-1} v \in \ker T$ and $V = \ker T \oplus \text{im }  T$, 

First to demonstrate that $\set v$ is linearly independent. 
$$T^{m-1} v = 0 \implies v \ne 0 \implies \set v \text{ is linearly independent}$$

Now to demonstrate that if $T^{m-1}v \ne 0$ and $T^m v = 0$ then .

%----------------------------------------------------------------------------------------
%	Problem 6
%----------------------------------------------------------------------------------------
\newpage
\section*{Problem 6}
\begin{problem}
Suppose $T \in \L{C^3}$ is defined by $T (z_1, z_2, z_3) = (z_2, z_3,0)$. Prove
that $T$ has no square root. More precisely, prove that there does not exist
$S \in \L{C^3}$ such that $S^2 = T$.
\end{problem}

\subsection*{Answer} .

%----------------------------------------------------------------------------------------
%	Problem 7
%----------------------------------------------------------------------------------------
\newpage
\section*{Problem 7}
\begin{problem}
Suppose $N \in \L V$ is nilpotent. Prove that $0$ is the only eigenvalue
of $N$.
\end{problem}

\subsection*{Answer} .

%----------------------------------------------------------------------------------------
%	Problem 9
%----------------------------------------------------------------------------------------
\newpage
\section*{Problem 9}
\begin{problem}
Suppose $S, T \in \L V$ and $S T$ is nilpotent. Prove that $T S$ is nilpotent.
\end{problem}
\subsection*{Answer} .


%----------------------------------------------------------------------------------------
%	Problem 12
%----------------------------------------------------------------------------------------
\newpage
\section*{Problem 12}
\begin{problem}
Suppose $N \in \L V$ and there exists a basis of $V$ with respect to which
$N$ has an upper-triangular matrix with only $0$’s on the diagonal. Prove
that $N$ is nilpotent.
\end{problem}
\subsection*{Answer} .


\end{document}
