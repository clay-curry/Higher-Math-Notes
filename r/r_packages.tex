% !TEX TS-program = pdflatex
% !TEX encoding = UTF-8 Unicode

\documentclass[11pt]{article} % use larger type; default would be 10pt
\usepackage[utf8]{inputenc} % set input encoding (not needed with XeLaTeX)
\usepackage{clays_notes}

% FONT STYLES

\title{R Packages \\ https://r-pkgs.org/intro.html}
\author{Notes by:  \\ Clay Curry}
\date{}

\begin{document}

\maketitle

The chances are that someone has already solved a problem that you're working on, and you can benefit from their work by downloading their package from the more than 14,000 packages available on the \textbf{C}omprehensive \textbf{R} \textbf{A}rchive \textbf{N}etwork.

\definition{R Packages}
{In R, the \textbf{package} is the fundamental unit of shareable code, bundling together \textit{code, data, documentation}, and \textit{tests}.}

Organising code in a package makes your life easier because R come with built-in conventions. For example,
\points
{you put R code in R/}
{you put tests in tests/}
{you put data in data/}


Anything that can be automated, should be automated. This philosophy is realised primarily through the devtools package, which is the public face for a suite of R functions that automate common development tasks. Devtools has recently (version 2.0.0 in October 2018) been restructured into a set of more focused packages, with devtools becoming more of a meta-package. The usethis package is the sub-package you are most likely to interact with directly; we explain the devtools-usethis relationship in section 3.2.

The 

\end{document}

















