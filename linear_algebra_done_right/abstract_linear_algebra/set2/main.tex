%%%%%%%%%%%%%%%%%%%%%%%%%%%%%%%%%%%%%%%%%
% fphw Assignment
% LaTeX Template
% Version 1.0 (27/04/2019)
%
% This template originates from:
% https://www.LaTeXTemplates.com
%
% Authors:
% Class by Felipe Portales-Oliva (f.portales.oliva@gmail.com) with template 
% content and modifications by Vel (vel@LaTeXTemplates.com)
%
% Template (this file) License:
% CC BY-NC-SA 3.0 (http://creativecommons.org/licenses/by-nc-sa/3.0/)
%
%%%%%%%%%%%%%%%%%%%%%%%%%%%%%%%%%%%%%%%%%

%----------------------------------------------------------------------------------------
%	PACKAGES AND OTHER DOCUMENT CONFIGURATIONS
%----------------------------------------------------------------------------------------

\documentclass[
	12pt, % Default font size, values between 10pt-12pt are allowed
	%letterpaper, % Uncomment for US letter paper size
	%spanish, % Uncomment for Spanish
]{fphw}

% Template-specific packages
\usepackage[utf8]{inputenc} % Required for inputting international characters
\usepackage[T1]{fontenc} % Output font encoding for international characters
\usepackage{mathpazo} % Use the Palatino font
\usepackage{graphicx} % Required for including images
\usepackage{booktabs} % Required for better horizontal rules in tables
\usepackage{listings} % Required for insertion of code
\usepackage{enumerate} % To modify the enumerate environment

\usepackage{amsmath}
\usepackage{amssymb}
%----------------------------------------------------------------------------------------
%	MY SHORTCUTS
%----------------------------------------------------------------------------------------

\newcommand\set[1]{\{#1\}}
\newcommand\qed{\text{$\blacksquare$}}
\newcommand\R[1]{\text{$\mathbb{R}^{#1}$}}
\newcommand\F[1]{\text{$\mathbb{F}^{#1}$}}
\newcommand\Z{\mathbb{Z}}
\newcommand\U{\text{$U$ }}
\newcommand\ls[2]{\text{$#1_1, \ldots, #1_{#2}$}}
\newcommand\poly[1]{\text{$\mathcal{P}_{#1}(\F{})$ }}
%----------------------------------------------------------------------------------------

%----------------------------------------------------------------------------------------
%	ASSIGNMENT INFORMATION
%----------------------------------------------------------------------------------------

\title{Homework \#1} % Assignment title
\author{Clayton Curry} % Student name
\date{Sep 5, 2021} % Due date
\institute{University of Oklahoma \\ Department of Mathematics} % Institute or school name
\class{Abstract Linear Algebra} % Course or class name
\professor{Dr. Gregory Muller} % Professor or teacher in charge of the assignment

%----------------------------------------------------------------------------------------

\begin{document}

\maketitle % Output the assignment title, created automatically using the information in the custom commands above

%----------------------------------------------------------------------------------------
%	ASSIGNMENT CONTENT
%----------------------------------------------------------------------------------------

\section*{1C: 7}

\begin{problem}
Give an example of a nonempty subset \U of \R{2} such that \U is closed under addition and under taking additive inverses (meaning $-u \in U$ whenever $u \in \U$ ), but \U is not a subspace of \R{2}.
\end{problem}

%------------------------------------------------

\subsection*{Answer} Will demonstrate by example:\\


%----------------------------------------------------------------------------------------

\section*{1C: 8}

\begin{problem}
Give an example of a nonempty subset \U of \R{2} such that \U is closed under scalar multiplication, but \U is not a subspace of \R{2}.
\end{problem}

%------------------------------------------------

\subsection*{Answer} Will demonstrate by example:\\


%----------------------------------------------------------------------------------------

\section*{1C: 9}

\begin{problem}
A function $f : \R{} \to \R{}$ is called periodic if there exists a positive number $p$ such that $f(x) = f(x + p)$ for all $x \in \R{}$. Is the set of periodic functions from \R{} to \R{} a subspace of \R{\R{}}? Explain.\end{problem}

%------------------------------------------------

\subsection*{Answer} Will demonstrate by example:\\


%----------------------------------------------------------------------------------------

\section*{2A: 1}

\begin{problem}
Suppose $v_1, v_2,  v_3, v_4$ spans $V$. Prove that the list
$$v_1-v_2, v_2 - v_3, v_3-v_4, v_4$$
also spans $V$.

\end{problem}

%------------------------------------------------

\subsection*{Answer} Will demonstrate by example:\\


%----------------------------------------------------------------------------------------

\section*{2A: 3}

\begin{problem}
Find a number $t$ such that 
$$(3,1,4), (2,-3,5), (5,9,t)$$
is not linearly independent in \R{3}.

\end{problem}

%------------------------------------------------

\subsection*{Answer} Will demonstrate by example:\\


%----------------------------------------------------------------------------------------

\section*{2A: 9}

\begin{problem}
Prove or give a counterexample: If \ls{v}{m} and \ls{w}{m} are linearly independent lists of vectors in $V$, then $v_1 + w_1 + \cdots + v_m + w_m$ is linearly independent.

\end{problem}

%------------------------------------------------

\subsection*{Answer} Will demonstrate by counter-example:\\


%----------------------------------------------------------------------------------------


\section*{2B: 5}

\begin{problem}
Prove or disprove: there exists a basis $p_0, p_1, p_2, p_3$ of \poly{3} such that none of the polynomials $p_0, p_1, p_2, p_3$ has degree 2.

\end{problem}

%------------------------------------------------

\subsection*{Answer} Will demonstrate by counter-example:\\


%----------------------------------------------------------------------------------------


\section*{2A: 6}

\begin{problem}
Suppose $v_1, v_2, v_3, v_4$ is a basis of $V$. Prove that
$$
v_1 + v_2, v_2 + v_3, v_3 + v_4, v_4
$$
is a basis of $V$.
\end{problem}

%------------------------------------------------

\subsection*{Answer} Will demonstrate by counter-example:\\


%----------------------------------------------------------------------------------------


\end{document}
