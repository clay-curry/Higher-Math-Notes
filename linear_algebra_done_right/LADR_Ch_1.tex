% !TEX TS-program = pdflatex
% !TEX encoding = UTF-8 Unicode

\documentclass[11pt]{article} % use larger type; default would be 10pt
\usepackage[utf8]{inputenc} % set input encoding (not needed with XeLaTeX)

%%% PAGE DIMENSIONS
\usepackage{geometry} % to change the page dimensions
\geometry{a4paper} % or letterpaper (US) or a5paper or....
 \geometry{margin=20mm} % for example, change the margins to 2 inches all round
\usepackage{myheader}


\title{\LARGE {Linear Algebra Done Right}}
\author{Sheldon Axler}
%\date{} % Activate to display a given date or no date (if empty),
         % otherwise the current date is printed 

\begin{document}
\maketitle

\section{Vector Spaces}

Linear algebra is the study of linear maps on finite-dimensional vector spaces. Vector spaces are defined in this chapter, and their basic properties are developed. Vector spaces are a generalization of the description of a plane using two coordinates, as published by Descartes in 1637. 

\subsection{$\R{n}$, $\C{n}$, and $\F{n}$}

\begin{deff}{Complex Number}

A \textbf{complex number} is an ordered pair $(a,b) \in \mathbf{R}^2$, denoted $a + bi$.
\begin{itemize}
\item The set of all complex numbers is denoted by $\mathbf{C}$:
  $$
    \mathbf{C} = \{a+bi : a,b \in \mathbf{R}\}
  $$
 \item \textbf{Addition and multiplication} on $\mathbf{C}$ are defined by
  $$
  (a + bi) + (c + di) \equiv (a + c) + (b + d)i 
  $$
  $$
  (a + bi)\cdot(c+di) \equiv (ac - bd) + (ad + bc)i
  $$
\end{itemize}
\end{deff}

\begin{thmm}{Complex Number}

A \textbf{complex number} is an ordered pair $(a,b) \in \mathbf{R}^2$, denoted $a + bi$.
\end{thmm}


\end{document}

















