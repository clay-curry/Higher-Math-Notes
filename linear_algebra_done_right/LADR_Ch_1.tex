% !TEX TS-program = pdflatex
% !TEX encoding = UTF-8 Unicode

\documentclass[11pt]{article} % use larger type; default would be 10pt
\usepackage[utf8]{inputenc} % set input encoding (not needed with XeLaTeX)
\usepackage{clays_notes}

% FONT STYLES

\title{Linear Algebra Done Right \\ Axler, Sheldon}
\author{Notes by:  \\ Clay Curry}
\date{}

\begin{document}

\maketitle
\tableofcontents
\clearpage

\section{Vector Spaces}

Linear algebra is the study of linear maps on finite-dimensional vector spaces. Vector spaces are defined in this chapter, and their basic properties are developed. Vector spaces are a generalization of the description of a plane using two coordinates, as published by Descartes in 1637. 

\subsection{$\R{n}$, $\C{n}$, and $\F{n}$}

\definition
{Complex Number}
{A \textbf{complex number} is an ordered pair $(a,b) \in \mathbf{R}^2$, denoted $a + bi$. 
	\points
	{
	The set of all complex numbers is denoted by $\C{ }$: 
	\mathdiv{\C{ } = \{a+bi : a,b \in \R{ } \}}
	}
	{
	\textbf{Addition and multiplication} on $\C{ }$ are defined by
	  \mathdiv{ (a + bi) + (c + di) \equiv (a + c) + (b + d)i }
	  \mathdiv{(a + bi)\cdot(c+di) \equiv (ac - bd) + (ad + bc)i }
	}
}



The following properies are proven using the familiar properites of real numbers and the definition of complex addition and multiplication.

\property{Properties of Complex Numbers}
{
\points
{\textbf{Commutativity : } $\alpha + \beta = \beta + \alpha$ and $\alpha\beta = \beta\alpha$ for all $\alpha, \beta \in \mathbf{C}$}
{\textbf{Associativity : } $(\alpha + \beta) + \lambda = \alpha + (\beta + \lambda)$ and $(\alpha \cdot \beta)\cdot \lambda = \alpha \cdot (\beta \cdot \lambda)$ for all  $\alpha, \beta, \lambda \in \mathbf{C}$}
{\textbf{Identities : } $\lambda + e_+ \equiv \lambda + 0 \equiv \lambda$ and $\lambda \cdot e_\cdot \equiv \lambda \cdot 1 \equiv \lambda$ for all $\lambda \in \mathbf{C}$}
{\textbf{Additive Inverse : } $\forall \alpha \in \mathbf{C}, \exists \beta \in \mathbf{C}$, such that $\alpha + \beta = e_+ = 0$}
{\textbf{Multiplicative Inverse : } $\forall \alpha \in \mathbf{C}, \exists \beta \in \mathbf{C}$, such that $\alpha \cdot \beta = e_\cdot = 1$}
{\textbf{Distributive Property} $\lambda \cdot (\alpha + \beta) = \lambda \cdot \alpha + \lambda \cdot \beta$, for all $\alpha, \beta, \lambda \in \mathbf{C}$}
}


\definition{Constructed Operations on $\C{}$: Subtraction and Division}
{
Let $\alpha, \beta \in \C{}$.
	\points
	{Let $-\alpha$ denote the additive inverse of $\alpha$.  Thus, $-\alpha$ is the unique element of $\C{}$ such that $\alpha+ (-\alpha) = 0$.}
	{Subtraction on $\C{}$ is defined by \mathdiv{\beta - \alpha = \beta + (-\alpha)}}
	{Let $(1/\alpha)$ denote the multiplicative inverse of $\alpha$.  Thus, $(1/\alpha)$ is the unique element of $\C{}$ such that $\alpha \cdot (1/\alpha) = 1$.}
	{Division on \textbf{C} is defined by \mathdiv{\beta/\alpha = \beta \cdot (1/\alpha)}}
	}


Throughout these notes, $\F{}$ stands for either $\R{}$ or $\C{}$. The letter $\F{}$ is used because $\R{}$ and $\C{}$ are examples of the algebraic structure known as a \textbf{field}.


%\definition{}{}

%\theorem{}{}{}

%\property{}{}


\end{document}

















