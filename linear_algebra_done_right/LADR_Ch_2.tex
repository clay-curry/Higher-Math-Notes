% !TEX TS-program = pdflatex
% !TEX encoding = UTF-8 Unicode

\documentclass[11pt]{article} % use larger type; default would be 10pt
\usepackage[utf8]{inputenc} % set input encoding (not needed with XeLaTeX)
\usepackage{clays_notes}


\newcommand\R[1]{\mathbf{R^{#1}}}
\newcommand\C[1]{\mathbf{C^{#1}}}
\newcommand\F[1]{\mathbf{F^{#1}}}
\newcommand\U{\mathbf{U}}
\newcommand\V{\mathbf{V}}
\newcommand\W{\mathbf{W}}
\newcommand\lc[3]{#1_1 #2_1 + \cdots + #1_{#3} #2_{#3}}
\newcommand\ls[2]{#1_1, \ldots, #1_{#2}}
\renewcommand\span[1]{\mathbf{span}(#1)}
\newcommand\set[1]{\{#1\}}
\newcommand\poly[1]{\mathcal{P}_{#1}(\F{})}
\renewcommand\deg{\text{deg }}
\renewcommand\dim{\text{dim}}


% FONT STYLES

\title{Linear Algebra Done Right \\ Axler, Sheldon}
\author{Notes by:  \\ Clay Curry}
\date{}

\begin{document}

\section{Finite-Dimensional Vector Spaces}
Not all vector spaces come in the same size. Ituitively, this makes sense given the feeling of having more room to move about in the space of $\R{3}$ than the mere plane of $\R{2}$. In fact, although $\R{2}$ and $\R{3}$ contain infinitely many points, it can be shown that, in the everyday sense of "size", the size of $\R{3}$ is indeed bigger than $\R{2}$ (some infinities are bigger than others). Linear algebra focues mainly on the properties of finite-dimensional vector spaces, which are introduced in this chapter. 

The basic tools needed to understand the structure of functions (or mappings) between finite-dimensional vector spaces are also defined in this chapter, namely span, linear independence, basis, and dimension. 

\subsection{Span and Linear Independence}
Adding up scalar multiples of vectors in a list gives what is called a \textbf{linear combination} of the list.

\subsubsection{Linear Combinations and Span}

\definition{Linear Combination}
{
A \textbf{linear combination} of a list $\ls{v}{m}$ of vectors in $\V$ is a vector of the form
\mathdiv{\lc{a}{v}{m}}
where $\ls{a}{m} \in \F{}$.
}

Some mathematicians use the term \textbf{linear span} or \textbf{linear hull}, which means the same thing as span. 

\definition{Span, Generating Set}
{
The set of all linear combinations of a list of vector $\ls{v}{m} \in \V$ is called the \textbf{span} of $\ls{v}{m}$, denoted $\span{\ls{v}{m}}$. In other words
\mathdiv{\span{\ls{v}{m}} = \set{\lc{a}{v}{m} : \ls{a}{m} \in \F{}}.}
To express that a vector space $\V$ is a span of a set $\ls{v}{m}$, one commonly uses the following phrases:
\points
{$\ls{v}{m}$ \textbf{spans} $\V$;}
{$\ls{v}{m}$ \textbf{generates} $\V$;}
{$\V$ is spanned by $\ls{v}{m}$;}
{$\V$ is generated by $\ls{v}{m}$;}
{$\ls{v}{m}$ is a spanning set of $\V$;}
{$\ls{v}{m}$ is a generating set of $\V$;}
The span of the empty list $()$ is defined to be zero.
}

The concept of span is particularly useful for describing the cardinality or "size" of a vector space. For dealing with notions related to size, the meaning of words like "larger" and "smaller" are defined in terms of subset relations. In other words, a set $\mathbf{A}$ is bigger than a set $\mathbf{B}$ implies that $\mathbf{B \subset A}$.
The substructure (subgroup, subring, subalgebra, subspace) generated by some list of elements in a mathematical structure is the smallest substructure containing at least those elements in the list.

\theorem{Span is the smallest containing subspace}
{
The span of a list of vectors in $V$ is the smallest subspace of $V$ containing all the vectors in the list.
}
{}
\newpage



\example{Natual Basis}
{
Suppose $n \in \mathbf{N}$. Show that
\mathdiv{(1, 0, \ldots, 0), (0, 1, 0, \ldots,0), \ldots, (0, \ldots, 0, 1)}
spans $\F{n}$. Here the $j^{\text{th}}$ vector in the list above is the $n$-tuple with 1 in the $j^{\text{th}}$ slot and $0$ in all the other slots.
\vspace{10 pt}
Suppose $(\ls{x}{n}) \in \F{n}$. Then
\mathdiv{(\ls{x}{n}) = (x_1, 0, \ldots, 0) + \cdots + (0, \ldots, 0, x_n) = x_1(1, 0, \ldots, 0) + \cdots + x_n(0, \ldots, 0, 1)}
Thus $(\ls{x}{n}) \in \span{(1, 0, \ldots, 0), \ldots, (0, \ldots, 0, 1)}$, as desired.
}

Now we can make one of the key definitions in linear algebra. Recall that every list, by definition, has a finite length.
\definition{Finite Dimensional Vector Space}
{A vector space is called \textbf{finite-dimensional} if some list of vectors in it spans the space.}
The above example shows that $\F{n}$ is a finite-dimensional vector space for every positive integer $n$.

\definition{Polynomial, $\poly{}$}
{
A function $p : \F{} \to \F{}$ is called a \textbf{polynomial} with coefficients in $\F{}$ if there exists $a_0, \ls{a}{m} \in \F{}$ such that
\mathdiv{p(z) = a_0 + a_1 z + a_2 z^2 + \cdots + a_m z^m}
for all $z \in \F{}$.
\points
{$\poly{}$ is the set of all polynomials with coefficients in $\F{}$.}
}

With the usual operations of addition and scalar multiplication, $\poly{}$ is a vector space over $\F{}$. As such, $\poly{}$ is a subspace of $\F{\F{}}$.

It will later be shown that the coefficients of the polynomial uniquely determine the polynomial.
\definition{degree of a polynomial, deg p}
{
\points
{A polynomial $p \in \poly{}$ is said to have \textit{\textbf{degree} m} if there exists scalars $a_0, \ls{a}{m} \in \F{}$ with $a_m \ne 0$ such that
\mathdiv{p(z) = a_0 + a_1 z + a_2 z^2 + \cdots + a_m z^m}
for all $z \in \F{}$. If $p$ has degree $m$, we write $\deg p = m$.}
{The polynomial that is identically 0 is said to have degree $- \infty$.}
}

We use the convention that $- \infty < m$, which means that the polynomial $0$ is in $\poly{m}$.
\definition{$\poly{m}$}
{For any $m \in \mathbb{N}$, $\poly{m}$ denotes the set of all polynomials with coefficients in $\F{}$ and degree at most $m$.}


\example{Finite-dimensional vector space}
{$\poly{m}$ is a finite dimensional vector space for each non-negative integer $m$.}

\definition{Infinite-dimensional vector space}
{A vector space is called \textbf{infinite-dimensional} if it is not finite dimensional.}

\example{Infinite-dimensional vector space}
{$\poly{}$ is infinite dimensional.}

\subsubsection{Linear Independence}
Suppose $\ls{v}{m} \in V$ and $v \in \span{\ls{v}{m}}$. By the definition of span, there exists $\ls{a}{m} \in \F{}$ such that
\mathdiv{v=\lc{a}{v}{m}.}

Consider the question of whether the choice of scalars in the expression is unique. Suppose $\ls{c}{n} \in \F{}$ is another set of scalars such that
\mathdiv{v = \lc{c}{v}{m}.}

Subtracting the last two equations, we have
\mathdiv{0 = (a_1 - c_1) v_1 + \cdots + (a_m - c_m) v_m.}

If the only way to write $0$ as a linear combination of $\ls{v}{m}$ is by taking $0$ for each scalar then, $a_j = c_j$ (the choice of scalars for each vector in $\span{\ls{v}{m}}$ must be unique). This situation is so important that we give it a special name, linear independence.

\definition{Linearly Independent}
{\points
{A list $\ls{v}{n}$ of vectors in $V$ is called \textbf{linearly independent} if the only choice of $\ls{a}{n} \in \F{}$ that makes $\lc{a}{v}{n}$ equal to $0$ is $a_1 = \cdots = a_n = 0$.}
{The empty list () is also declared to be linearly independent.}
}

\definition{Linearly dependent}
{\points{A list $\ls{v}{n}$ of vectors in $V$ is called \textbf{linearly dependent} if it is not linearly independent.}}


\newpage
\theorem{Linear Dependence Lemma}
{
Suppose $\ls{v}{m}$ is a linearly dependent list in $V$. Then there exists $j \le m$ such that the following hold:
\points
{$v_j \in \span{\ls{v}{j-1}}$;}
{If the $j^{\text{th}}$ term is removed from $\ls{v}{m}$, the span of the remaining list equals $\span{\ls{v}{m}}$.}
}
{}

\theorem{Length of a linearly independent list $\le$ length of spanning list}
{
In a finite-dimensional vector space, the length of every linearly indepenedent list of vectors is less than or equal to the length of every spanning list of vectors.
}
{}

\theorem{Finite dimensional subspaces}
{Every subspace of a finite-dimensional vector space is finite-dimensional}
{}

\newpage

\subsection{Bases}
The following section derives crucial theorems built on concepts of linear independence and span.

\definition{Basis}
{A \textbf{basis} of $\V$ is a list of vectors in $\V$ that is linearly independent and spans $\V$.}

The next result explains why bases are useful. 

\theorem{Criterion for basis}
{A list $\ls{v}{n}$ of vectors in $\V$ is a basis of $\V$ if and only if every $v \in \V$ can be written uniquely in the form
\mathdiv{v = \lc{a}{v}{n}}
where $\ls{a}{n} \in \F{}$
}
{}

\theorem{Spanning list contains a basis}
{Every spanning list in a vector space can be reduced to a basis of the vector space.}
{}

\theorem{Basis of finite-dimensional vector space}
{Every finite-dimensional vector space has a basis}
{}

\theorem{Linearly independent list extends to a basis}
{Every linearly independent list of vectors in a finite-dimensional vector space can be extended to a basis of the vector space.}
{}

\theorem{Every subspace of $\V$ is part of a direct sum equal to $\V$}
{Suppose $\V$ is finite-dimensional and $\U$ is a subspace of $\V$. Then there is a subspace $\W$ of $\V$ such that $\V = \U \oplus \W$.}
{}


\subsection{Dimension}

\theorem{Basis length does not depend on basis}
{Any two bases of a finite-dimensional vector space have the same length.}
{}

\definition{Dimension, $\dim V$}
{\points
{The \textbf{dimension} of a finite-dimensional vector space is the length of any basis of the vector space.}
{The dimension of $V$ (if $V$ is finite-dimensional) is denoted by $\dim{V}$.}
}

\theorem{Dimension of a subspace}
{If $\V$ is finite-dimensional and $\U$ is a subspace of $\V$, then $\dim{\U} \le \dim{\V}$.}
{}

\theorem{Linearly independent list of the right length is a basis}
{Suppose $\V$ is finite-dimensional. Then every linearly indepenedent list of vectors in $\V$ with length $\dim{\V}$ is a basis of $\V$.}
{}

\theorem{Spanning list of the right length is a basis}
{Suppose $\V$ is finite-dimensional. Then every spanning list of vectors in $\V$ with length $\dim{\V}$ is a basis of $\V$.}
{}

\theorem{Dimension of a sum}
{If $U_1$ and $U_2$ are subspaces of a finite dimensional vector space, then
\mathdiv{\dim(U_1 + U_2) = \dim U_1 + \dim U_2 - \dim (U_1 \cap U_2)}
}
{}

\end{document}