% !TEX TS-program = pdflatex
% !TEX encoding = UTF-8 Unicode

\documentclass[11pt]{article} % use larger type; default would be 10pt
\usepackage[utf8]{inputenc} % set input encoding (not needed with XeLaTeX)
\usepackage{clays_notes}


\newcommand\R[1]{\mathbf{R^{#1}}}
\newcommand\C[1]{\mathbf{C^{#1}}}
\newcommand\F[1]{\mathbf{F^{#1}}}
\renewcommand\L{\mathcal{L}}

\newcommand\lc[3]{#1_1 #2_1 + \cdots + #1_{#3} #2_{#3}}
\newcommand\ls[2]{#1_1, \ldots, #1_{#2}}
\renewcommand\span[1]{\mathbf{span}(#1)}
\newcommand\set[1]{\{#1\}}
\newcommand\poly[1]{\mathcal{P}_{#1}(\F{})}
\renewcommand\deg{\text{deg }}
\renewcommand\dim{\text{dim}}


% FONT STYLES

\title{Linear Algebra Done Right \\ Axler, Sheldon}
\author{Notes by:  \\ Clay Curry}
\date{}

\begin{document}

\section{Linear Maps}
\subsection{The Vector Space of Linear Maps}
\definition{Linear Map}
{
A \textbf{linear map} from $V$ to $W$ is a function $T : V \to W$ with the following properies:
\points
{\textbf{Additivity: } $T(u+v) = Tu + Tv$ for all $u, v \in V$;}
{\textbf{Homogeneity: } $T(\lambda v) = \lambda Tv$ for all $\lambda \in \F{}$ and $v \in V$.}
The set of all linear maps from $V$ to $W$ is denoted $\L(V,W)$.
}


\end{document}