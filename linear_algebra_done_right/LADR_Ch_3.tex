% !TEX TS-program = pdflatex
% !TEX encoding = UTF-8 Unicode

\documentclass[11pt]{article} % use larger type; default would be 10pt
\usepackage[utf8]{inputenc} % set input encoding (not needed with XeLaTeX)
\usepackage{clays_notes}


\newcommand\R[1]{\mathbf{R^{#1}}}
\newcommand\C[1]{\mathbf{C^{#1}}}
\newcommand\F[1]{\mathbf{F^{#1}}}
\renewcommand\L[2]{\mathcal{L}(#1,#2)}

\newcommand\lc[3]{#1_1 #2_1 + \cdots + #1_{#3} #2_{#3}}
\newcommand\ls[2]{#1_1, \ldots, #1_{#2}}
\renewcommand\span[1]{\mathbf{span}(#1)}
\newcommand\set[1]{\{#1\}}
\newcommand\poly[1]{\mathcal{P}_{#1}(\F{})}
\renewcommand\deg{\text{deg }}
\renewcommand\dim{\text{dim }}
\renewcommand\null{\text{null }}
\newcommand\range{\text{range }}


% FONT STYLES

\title{Linear Algebra Done Right \\ Axler, Sheldon}
\author{Notes by:  \\ Clay Curry}
\date{}

\begin{document}

\section{Linear Maps}
\subsection{The Vector Space of Linear Maps}
\subsubsection{Definition and Examples of Linear Maps}
\definition{Linear Map}
{
A \textbf{linear map} from $V$ to $W$ is a function $T : V \to W$ with the following properies:
\points
{\textbf{Additivity: } $T(u+v) = Tu + Tv$ for all $u, v \in V$;}
{\textbf{Homogeneity: } $T(\lambda v) = \lambda Tv$ for all $\lambda \in \F{}$ and $v \in V$.}
The set of all linear maps from $V$ to $W$ is denoted $\L{V}{W}$.
}

\theorem{Linear maps and basis of domain}
{Suppose $\ls{v}{n}$ is a basis of $V$ and $\ls{w}{n} \in W$. Then there exists a unique linear map $T: V \to W$ such that \mathdiv{Tv_j = w_j} for each $j =1, \ldots, n$.}
{}

\subsubsection{Algebraic Operations on $\L{V}{W}$}
\definition{Addition and Scalar Multiplication on $\L{V}{W}$}
{
Suppose $S, T \in \L{V}{W}$ and $\lambda \in \F{}$. The \textbf{sum} $S+T$ and the \textbf{product} $\lambda T$ are the maps from $V$ to $W$ defined by:
\mathdiv{(S+T)(v) = Sv + Tv \text{ \hspace{5 pt} and \hspace{5 pt} } (\lambda T)(v) = \lambda (Tv)}
for all $v \in V$.
}

\theorem{$\L{V}{W}$ is a vector space}
{With the operations of addition and scalar multiplication as defined above, $\L{V}{W}$ is a vector space.}
{}

\definition{Product of Linear Maps}
{If $T \in \L{U}{V}$ and $S \in \L{V}{W}$, then the \textbf{product} $ST : U \to W$ is defined by \mathdiv{(ST)(u) = S(Tu)}for $u \in U$.}

\theorem{The product of linear maps is a linear map}
{If $T \in \mathcal{L}(U,V)$ and $S \in \L{V}{W}$, then $ST \in \L{U}{W}$.}
{}

\property{Algebraic properties of products of linear maps}
{
\points
{\textbf{Associativity: }\mathdiv{(T_1T_2)T_3 = T_1(T_2T_3)}whenever $T_1, T_2,$ and $T_3$ are linear maps such that products make sense.}
{\textbf{Identity: }\mathdiv{TI = IT = T}whenever $T \in \L{V}{W}$ and the left $I$ is the identity map on $V$ and the right $I$ is the identity map on $W$.}
{\textbf{Distributitvity: }\mathdiv{(S_1 + S_2)T = S_1T + S_2T \text{ \hspace{5pt} and \hspace{5pt} } S(T_1 + T_2) = ST_1 + ST_2}whenever $S, S_1, S_2 \in \L{V}{W}$ and $T, T_1, T_2 \in \L{U}{V}$.}
}

\theorem{Linear maps take 0 to 0}
{Suppose $T \in \L{V}{W}$. Then $T(0) = 0$.}
{}


%%%%%%%%%%%%%%%%%%%
% Section 3.2 : Null Spaces and Ranges
%%%%%%%%%%%%%%%%%%%

\subsection{Null Spaces and Ranges}
\subsubsection{Null Spaces and Injectivity}

In this section, we will learn about two subspaces intimitely connected with each linear map. The first subspace exists in the domain; the second exists in the codomain.

\definition{Null Space}
{For $T \in \L{V}{W}$, the \textbf{null space} of $T$, denoted $\null T$, is the subset of $V$ consisting of all the vectors $v \in V$ that $T$ maps to $0 \in W$:\mathdiv{\null T = \set{v \in V : Tv = 0 \in W}}}

Some mathematicians use the term \textbf{kernel} instead of null space.

\theorem{The null space is a subspace}
{Suppose $T \in \L{V,W}$. Then $\null T$ is a subspace of $V$.}
{}

\definition{Injective, one-to-one}
{A function $T : V \to W$ is called injective if $Tu = Tv$ implies $u=v$.}

\theorem{Injectivity is equivalent to null space equals $\set{0}$}
{Let $T \in \L{V,W}$. Then $T$ is injective if and only if $\null T = \set{0}$}
{}

\subsubsection{Range and Surjectivity}
\definition{Range}
{For a function $T : V \to W$, the range of $T$ is the subset of $W$ consisting of those vectors that are of the form $Tv$ for some $v \in V$:
\mathdiv{\range T = \set{Tv : v \in V}}
}

\theorem{The range is a subspace}
{Suppose $T \in \L{V}{W}$. Then $\range T$ is a subspace of $W$.}
{}

\definition{Surgective, onto}
{A function $T : V \to W$ is called \textbf{surjective} or \textbf{onto} if its range is equal to $W$.}

\subsubsection{Fundamental Theorem of Linear Maps}
The next result is so important that it gets a dramatic name.
\theorem{Fundamental Theorem of Linear Maps}
{Suppose $V$ is finite-dimensional and $T \in \L{V}{W}$. Then $\range T$ is finite-dimensional and \mathdiv{\dim V = \dim \null T + \dim \range T}}
{}

Now we can show that no linear map from a finite-dimensional vector space to a "smaller" vector space can be injective, where "smaller" is measured in dimension.

\theorem{A map to a smaller dimensional subspace is not injective}
{Suppose $V$ and $W$ are finite-dimensional vector spaces such that $\dim V > \dim W$. Then no linear map from $V$ to $W$ is injective.}
{}

\theorem{A map to a larger dimensional subspace is not surjective}
{Suppose $V$ and $W$ are finite-dimensional vector spaces such that $\dim V < \dim W$. Then no linear map from $V$ to $W$ is surjective.}
{}

\theorem{Homogenous system of linear equations}
{A homogenous system of linear equations with more variables than eqautions has non-zero solutions.}
{}

\theorem{Inhomogeneous system of linear equations}
{An inhomogeneous system of linear equations with more equations than variables has no solution for some choice of the constant terms.}
{}

%%%%%%%%%%%%%%%%%%%
% Section 3.2 : Matrices
%%%%%%%%%%%%%%%%%%%

\end{document}