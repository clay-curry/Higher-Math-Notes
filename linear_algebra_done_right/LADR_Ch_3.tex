% !TEX TS-program = pdflatex
% !TEX encoding = UTF-8 Unicode

\documentclass[11pt]{article} % use larger type; default would be 10pt
\usepackage[utf8]{inputenc} % set input encoding (not needed with XeLaTeX)
\usepackage{clays_notes}


\newcommand\R[1]{\mathbf{R^{#1}}}
\newcommand\C[1]{\mathbf{C^{#1}}}
\newcommand\F[1]{\mathbf{F^{#1}}}
\renewcommand\L[2]{\mathcal{L}(#1,#2)}

\newcommand\lc[3]{#1_1 #2_1 + \cdots + #1_{#3} #2_{#3}}
\newcommand\ls[2]{#1_1, \ldots, #1_{#2}}
\renewcommand\span[1]{\mathbf{span}(#1)}
\newcommand\set[1]{\{#1\}}
\newcommand\poly[1]{\mathcal{P}_{#1}(\F{})}
\renewcommand\deg{\text{deg }}
\renewcommand\dim{\text{dim}}


% FONT STYLES

\title{Linear Algebra Done Right \\ Axler, Sheldon}
\author{Notes by:  \\ Clay Curry}
\date{}

\begin{document}

\section{Linear Maps}
\subsection{The Vector Space of Linear Maps}
\subsubsection{Definition and Examples of Linear Maps}
\definition{Linear Map}
{
A \textbf{linear map} from $V$ to $W$ is a function $T : V \to W$ with the following properies:
\points
{\textbf{Additivity: } $T(u+v) = Tu + Tv$ for all $u, v \in V$;}
{\textbf{Homogeneity: } $T(\lambda v) = \lambda Tv$ for all $\lambda \in \F{}$ and $v \in V$.}
The set of all linear maps from $V$ to $W$ is denoted $\L{V}{W}$.
}

\theorem{Linear maps and basis of domain}
{Suppose $\ls{v}{n}$ is a basis of $V$ and $\ls{w}{n} \in W$. Then there exists a unique linear map $T: V \to W$ such that \mathdiv{Tv_j = w_j} for each $j =1, \ldots, n$.}
{}

\subsubsection{Algebraic Operations on $\L{V}{W}$}
\definition{Addition and Scalar Multiplication on $\L{V}{W}$}
{
Suppose $S, T \in \L{V}{W}$ and $\lambda \in \F{}$. The \textbf{sum} $S+T$ and the \textbf{product} $\lambda T$ are the maps from $V$ to $W$ defined by:
\mathdiv{(S+T)(v) = Sv + Tv \text{ \hspace{5 pt} and \hspace{5 pt} } (\lambda T)(v) = \lambda (Tv)}
for all $v \in V$.
}

\theorem{$\L{V}{W}$ is a vector space}
{With the operations of addition and scalar multiplication as defined above, $\L{V}{W}$ is a vector space.}
{}

\definition{Product of Linear Maps}
{If $T \in \L{U}{V}$ and $S \in \L{V}{W}$, then the \textbf{product} $ST : U \to W$ is defined by \mathdiv{(ST)(u) = S(Tu)}for $u \in U$.}

\theorem{The product of linear maps is a linear map}
{If $T \in \mathcal{L}(U,V)$ and $S \in \L{V}{W}$, then $ST \in \L{U}{W}$.}
{}

\property{Algebraic properties of products of linear maps}
{
\points
{\textbf{Associativity: }\mathdiv{(T_1T_2)T_3 = T_1(T_2T_3)}whenever $T_1, T_2,$ and $T_3$ are linear maps such that products make sense.}
{\textbf{Identity: }\mathdiv{TI = IT = T}whenever $T \in \L{V}{W}$ and the left $I$ is the identity map on $V$ and the right $I$ is the identity map on $W$.}
{\textbf{Distributitvity: }\mathdiv{(S_1 + S_2)T = S_1T + S_2T \text{ \hspace{5pt} and \hspace{5pt} } S(T_1 + T_2) = ST_1 + ST_2}whenever $S, S_1, S_2 \in \L{V}{W}$ and $T, T_1, T_2 \in \L{U}{V}$.}
}

\theorem{Linear maps take 0 to 0}
{Suppose $T \in \L{V}{W}$. Then $T(0) = 0$.}
{}

\end{document}